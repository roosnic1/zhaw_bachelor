% protocol.tex - Bachelorthesis Design Review Protocol
% Author: Nicolas Roos

\documentclass[a4paper]{article}
\usepackage[ngerman]{babel}
\usepackage[utf8]{inputenc}
%\usepackage[T1]{fontenc}


\begin{document}

\title{Bachelorthesis Design-Review Protokoll}
%\subtitle{Verbesserung der Benutzer Erfahrung der Kundschaft eines internationalen Velokuriers}
\author{Nicolas Roos}
\date{12.06.2016}
\maketitle

\section{Informationen}

Titel: Verbesserung der Benutzer Erfahrung der Kundschaft eines internationalen Velokuriers
\newline{}
\newline{}
Anwesend: Reto Knaack, Beat Seeliger, Jaime Oberle, Nicolas Roos

\section{Design-Review Fragen}

\begin{enumerate}
\item Ist der Auftrag für die Bachelorarbeit von der Studentin oder dem Studenten korrekt erfasst? \\ \textbf{Ja}
\item Sind die Ausgangslage und das Umfeld ausreichend analysiert, berücksichtigt und bearbeitet? \\ \textbf{Ja}
\item Wurde eine systematische Recherche durchgeführt? \\ \textbf{Ja}
\item Wurde ein klares Konzept erarbeitet und klar dargestellt? \\ \textbf{Ja}
\item Wurden alternative Lösungen betrachtet? \\ \textbf{Ja}
\item Entsprechen das Arbeits- und das Lösungskonzept den Anforderungen an eine Bachelorarbeit? \\ \textbf{Ja}
\item Ist die Lösung grundsätzlich für die Auftraggeberin bzw. den Auftraggeber akzeptabel? \\ \textbf{Ja}
\item Ist das Konzept bzw. die Lösung technisch und terminlich im Rahmen der verbleibenden Zeit umsetzbar? \\ \textbf{Ja}
\item Sind die nächsten Arbeitsschritte klar formuliert? \\ \textbf{ja}
\end{enumerate}

\section{Weiteres}

Der späteste Zeitpunkt für die Abgabe der Arbeit ist am 17. Juli 2016.  Die Präsentation der Arbeit wird im Herbstsemester 2016/2017 stattfinden.

\end{document}
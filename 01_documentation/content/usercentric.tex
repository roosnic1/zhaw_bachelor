%!TEX root = ../doc.tex
\chapter{User Centered Design}
\label{sec:usercentric}
Als \textit{User Centered Design} werden Prinzipien und Werkzeuge verstanden, welche den Designern, Entwicklern und allen anderen, die an der Entstehung eines Produktes, Services oder Prozesses beteiligt sind, helfen, die Bedürfnisse und Wünsche des Endbenutzer während der gesamten Entwicklungsphase mit hoher Priorität in den Vordergrund zu stellen. Dieses Framework steht in starkem Kontrast zu herkömmlichen Entwicklungsmethoden, bei welchen die Stakeholders, zu einem grossen Teil die Ausrichtung, das Aussehen und Verhalten einer neuen Erschaffung bestimmen. Bei diesen älteren Entwicklungsmethoden kam es vermehrt dazu, dass ein Produkt erst bei der Veröffentlichung zum ersten Mal die Hände eines Endbenutzer berührte und sich dann als komplett unnütz herausstellte. Der \textit{User Centered Design}-Ansatz involviert mögliche Endbenutzer bereits ab den ersten Entwicklungsphasen und fordert das Zielpublikum zur Mitgestaltung auf. Dadurch werden viele Probleme, die bei einer Markteinführung auftreten können, bereits früh entdeckt und behoben. Die Anwendung dieses Frameworks kostet im Gegensatz zu anderen Methoden mehr Zeit und dadurch auch mehr Geld. \textit{User Centered Design} ist eine iterative Methode, welche nach der Durchführung ihrer Teilschritte wieder von vorne beginnt. Abbildung \ref{fig:usercentereddesign} zeigt eine mögliche schematische Darstellung des Ablaufs von \textit{User Centered Design}.

\begin{figure}[ht]
	\centering
  \includegraphics[width=0.88\textwidth]{images/userCenteredDesign.png}
	\caption{Prozessablauf von User Centered Design}
	\label{fig:usercentereddesign}
\end{figure}

Während all diesen Teilschritten werden die Ergebnisse und Erkenntnisse den möglichen Endbenutzern vorgestellt. Dabei wird überprüft, ob z.B. der Interaktionsfluss oder die Benutzeroberflächen einer Applikation verständlich sind oder alle relevanten Spezialfälle abdecken. \textit{User Centered Design} kann auf viele verschiedene Arte angewendet werden und zu den gewünschten Resultaten führen. Diese Bachelorarbeit orientiert sich am User Centered Design-Prozess, welcher im Buch von Kim Goodwin mit dem Namen \textit{Designing for the Digital Age} \citep{goodwin2011designing} beschrieben wird.


\section{Anwendung}
In den ersten Gesprächen mit den Stakeholdern wurde schnell klar, dass das Ziel des Produktes eine Webapplikation sein soll, die von Benutzern jeglicher Generation und Demographie benutzt werden kann. Die Entscheidung für eine Entwicklung unter \textit{User Centered Design} war naheliegend. Im Folgenden ist der geplante Ablauf mit dem Fokus auf den Endbenutzer beschrieben.

\subsection{Team}
Bei der Zusammenstellung des Design Teams ist es wichtig die verschiedenen benötigten Fähigkeiten in einer kleinen Gruppe zusammenzufassen. Ein Interaktionsdesigner hat einen ganz anderen Fokus als der grafische Designer. Im Rahmen dieser Bachelorarbeit fielen alle Rollen des Designs auf eine Person. Diese Konstellation ist nicht per se nachteilig. Die einzelne Person muss aber in der Lage sein, während den verschiedenen Phasen die Probleme, Ideen, Vorschläge, Lösungen und vieles andere aus den jeweiligen Perspektiven zu betrachten und beurteilen. In der idealen Zusammensetzung nach Goodwin sind folgende Rollen zu besetzen \citep[Kapitel 2]{goodwin2011designing}.
\begin{description}
	\item [Teamleiter\_in] Verantwortlich für das Team und die Koordination mit allen anderen Involvierten.
	\item [Interaction Designer\_in (Generator)] Verantwortlich für die Visualisierung des Systemverhaltens.
	\item [Interaction Designer\_in (Synthesizer)] Verantwortlich für die Analyse und Kommunikation des Designs.
	\item [Grafischer Designer\_in] Verantwortlich für die visuelle Umsetzung des Designs.
	\item [Industrie Designer\_in] Verantwortlich für das Design der Hardware.
\end{description}

\subsection{Grundlegende Recherche der Domäne}
Während der Recherche wird die Domäne, in der sich das zu erschaffende Produkt befindet, analysiert. Dabei werden Marktanalysen getätigt sowie falls vorhanden die Produkte der Konkurrenz untersucht. Zusätzlich werden Experten und Menschen, welche bereits lange in dem Bereich tätig sind, befragt. Diese Arbeit wird vom gesamten Design-Team bewältigt, da bereits in dieser Phase die verschiedenen Perspektiven wichtig sind. Mit Nick Blake als Gründer von Imagine Cargo und Experte der Kurier-, Express- und Paketdienstindustrie hat diese Bachelorarbeit auf sehr viel Erfahrung und Insiderwissen zurückgreifen können.

\subsection{Interviews mit Stakeholders}
Bevor die Interviews beginnen können, müssen die Stakeholder definiert werden. Zu den Stakeholdern gehören grundsätzlich alle Menschen, die ein Produkt mitbestimmen können. Jedoch zählt nicht jede Meinung gleich viel. Bei der Analyse ist es wichtig, mit allen involvierten Menschen zu reden und auch jene Personen ausfindig machen, welche sich aus dem ganzen Design-Prozess raushalten, aber die Macht haben das ganze Projekt zu beerdigen. Üblicherweise finden sich solche Stakeholder im Management. Es sind Personen, die das Gefühl haben ihre Meinung sei noch nicht notwendig beziehungsweise er/sie werde erst aktiv wenn das Projekt ein Fundament hat. Meist ist es dann aber zu spät für grundlegende Designänderungen. Für die Bachelorthesis wurden David Emmerth und Nick Blake von Imagine Cargo interviewt. Die Fragen für die Interviews sind vom Buch von Kim Goodwin inspiriert worden \citep[Kapitel 5]{goodwin2011designing} und sind im Anhang \ref{subsec:stakeholderfragen} zu finden.

\subsection{Interviews mit Endbenutzer}
In den Interviews mit den Stakeholdern sollte sich eine grobe Definition des zu erstellenden Produktes heraus kristallisieren. Mit dieser gewonnen Definition werden mit möglichen Endbenutzern Interviews geführt. Das soll weitere Einblicke in die Anforderungen und Unzufriedenheiten mit bestehenden Produkten zu Tage bringen. Aufgrund der beschränkten Ressourcen im Rahmen dieser Bachelorarbeit, wurde auf diesen Schritt verzichtet. Dafür sollen in der Visualisierungsphase die erstellten Wireframes und die Interaktionsabläufe mit möglichen Endbenutzern überprüft werden.

\subsection{Personas}
Aus den Interviews mit Stakeholdern und Endbenutzern, der Recherche und Gesprächen mit Experten werden sogenannte Personas erstellt. Eine Persona ist eine mit Fliesstext und Legende beschriebene Persönlichkeit. Im Text wird beschrieben wie sich die Person in Bezug auf das zu entwickelnde Produkt verhält und was ihre Voraussetzungen sind. Auch ihr Grundwissen wird definiert. Die Persona soll dem Design-Team ermöglichen, die Perspektive der beschriebenen Person einzunehmen und wie sie zu denken. Aufgrund der fehlenden Interviews mit Endbenutzer wurden für die Bachelorarbeit die Personas aufgrund der Interviews mit den Stakeholdern sowie der Recherche erstellt. Persönliche Erfahrungen mit Geschäftsbranchen, welche Kuriere nutzten, sowie Kontakte zu Fahrradkurieren haben zur Erstellung der Personas beigetragen. Die Personas sind im Kapitel \ref{sec:personas} definiert.

\subsection{Stories erstellen und Anforderungen definieren}
Aus den definierten Personas und den Interviews mit den Stakeholdern werden Persona Stories erstellt, welche die Anforderungen an das zu entwickelnde Produkt beschreiben. Die Persona Stories werden aus Sicht der Personas geschrieben. Diese grammatikalische Bedingung hilft den Autoren sich in die Gedanken, Gefühle und Denkweisen der Persona zu versetzten. Die Persona Stories für die Bachelorthesis sind in Kapitel \ref{sec:personastories} nach zu lesen.

\subsection{Wireframes und Interaktionsabläufe}
Die Personas, ihre Stories und die Anforderungen dienen als Grundlage für die Erstellung erster Wireframes\footnote{ Wireframe (Drahtmodell) bezeichnet eine visuelle Darstellung einer Benutzeroberfläche wobei der Fokus auf der Anordnung der Elemente und nicht auf der Gestaltung liegt} und Definitionen der Interaktionsabläufe. Wireframes verbunden mit Interaktionsabläufen sind schneller zu erzeugen als ein Prototyp zu entwickeln. Dadurch sind sie geeignet für die \textit{User Centered Design} Methode, weil der Benutzer dadurch in der Lage ist, das bisher theoretisch Besprochene praktisch zu sehen und es sich besser vorzustellen. Dadurch können erneut viele Missverständnisse zwischen dem Team und dem Endbenutzer beseitigt werden, dies alles vor der Entwicklung eines ersten Prototypen. Im Rahmen dieser Bachelorthesis wurden auf der Suche nach möglichen Endbenutzern 40 Zürcher Unternehmen angeschrieben. Weil die Kundschaft von ImagineCargo zum Zeitpunkt dieses Schrittes noch nicht existierte, wurden Architekturbüros, Werbeagenturen (mit Fokus auf Print), Anwaltskanzleien und Medizinlabors angeschrieben. Die Annahme dass diese Firmen regelmässig Kurier-, Express- und Paketdienstleistungen in Anspruch nehemen wurden, lag nahe. Von den knapp 40 angeschrieben Unternehmen antwortete jedoch nur acht mit der jeweils gleichen Antwort, sie sagten dass Sie zu unregelmässig KEP-Dienstleistungen benutzten würden und daher keine relevante und signifikante Hilfe sein können. Aufgrund der beschränkten Zeitdauer dieser Bachelorthesis wurden für die Überprüfung der Wireframes erneut die Stakeholder kontaktiert. Zusätzlich wurde vereinbart, dass nach der Veröffentlichung der ersten Version die effektiven Benutzer der Webapplikation für eine Überprüfung des Designs und der Interkationsabläufe kontaktiert werden. Die Wireframes sind im Anhang \ref{sec:anhangwireframes} zu finden.

\subsection{Prototyp}
Die Entwicklung des Prototyps ermöglicht es, die Machbarkeit zu überprüfen. Wie die restlichen Schritte findet auch die Entwicklung des Prototypen iterativ statt. Dadurch kann der Fortschritt regelmässig überprüft und den möglichen Endbenutzern zum ausprobieren vorgelegt werden. Dabei können schlecht oder nicht funktionierende Aktionen oder Interaktionsabläufe angepasst werden. Die Entwicklung des Prototypen im Rahmen dieser Bachelorthesis wurde in regelmässigen Abständen mit den Stakeholdern überprüft.


\section{Interviews}
\label{sec:interviews}

Die Interviews mit den Stakeholdern wurden in entspannter und informeller Atmosphäre einzeln geführt. Das Interview wurde mit einem Mobiltelefon aufgenommen, damit der Interviewer während des Interviews zu 100\% auf die Gesprächspartner konzentriert war. Diese Art von Interview lässt auch Abschweifungen zu, was nützliche Informationen zu Tage brachte, welche sonst nicht zur Sprache gekommen wären. Die Interviews sind im Anhang \ref{subsec:interviewzusammen} zusammengefasst.
%!TEX root = ../doc.tex
\chapter{Anforerdungsanalyse}
\label{sec:anforderungsanalyse}


\section{Einleitung}
Die Anforderungen werden mithilfe von User Stories in der Form \glqq Als <Rolle> möchte ich <Ziel/Wunsch>, um <Nutzen>\grqq festgehalten.

\section{Rollen}
\subsection{End User (EU)}
\textcolor{darkgray}{
	Den End User detailiert beschreiben
}
\subsection{System Administrator (SA)}
\textcolor{darkgray}{
	System Admin of Lobo?
}

\section{User Stories}
\subsection{End User (EU)}


\addcontentsline{toc}{subsection}{SC-001 Routen/Produkte anzeigen}
\begin{usecase}
\addtitle{SC-001}{Routen/Produkte anzeigen}
\addscenario{Story}{
	\item Als Kunde möchte ich Startort, Zielort und eine Uhrzeit eingeben können, um die verfügbaren Routen/Produkte zu sehen.
}
\addscenario{Akzeptanzkriterien}{
	\item Der Kunde sieht Routen welche zu einer bestimmten Uhrzeit zwischen seinem Start und Zielort liegen.
	\item ...
}
\addfield{Priorität}{Hoch}
\addfield{Domain}{???}
\end{usecase}

\newpage
\addcontentsline{toc}{subsection}{SC-002 Leistung kaufen}
\begin{usecase}
\addtitle{SC-002}{Leistung kaufen}
\addscenario{Story}{
	\item Als Kunde möchte ich eine Leistung kaufen können, um den Auftrag abzuschliessen.
}
\addscenario{Akzeptanzkriterien}{
	\item Der Kunde kann die Leistung kaufen, wenn sie fertig konfiguriert ist und ein akzeptiertes Zahlungsmittel eingegeben wurde.
	\item Der Kaufauftrag muss nach dem abschliessen persistent gespeichert sein und ein Abbuchungsbefehl beim Zahlungsinstitut aufgegeben sein.
	\item Der Kunde bekommt per Mail eine Auftragsbestätigung.
}
\addfield{Priorität}{Hoch}
\addfield{Domain}{???}
\end{usecase}


\section{Akzeptanztests}

\begin{description}
\item[US-1] End User sieht nach Eingabe des Start- und Zielort alle verfügbaren Routen.
\end{description}


\section{Priorisierung}
Die User Stories werden mit dem MoSCoW Prinzip priorisiert.
???




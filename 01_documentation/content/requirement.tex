%!TEX root = ../doc.tex
\chapter{Anforerdungsanalyse}
\label{sec:anforderungsanalyse}


\section{Einleitung}
Mit dem User Centered Design Prozess konzentriert sich die Anforderungsanalyse sehr stark auf den Benutzer. Dennoch wurden mit den wichtigsten Stakeholders Interviews geführt, in welchen sich zeigte was für Aufgaben und die Webapplikation für die Firma lösen soll.




\section{Stakeholders}
\begin{itemize}
	\item David Emmerth
	\item Nick Blake
	\item Jürgen (Lobo)
\end{itemize}

\subsection{Zusammenfassung der Interviews}

\section{Personas}

\subsection{Einleitung}
Personas sind Urbilder welche die verschiedenen Ziele und Verhaltensmuster der typischen Benutzer beschreiben sollen. Sie unterscheiden sich von anderen Methoden hauptsächlich durch ihre Geschichte erzählenden Art, welche bei Entwicklern und Designern besonders die sozialen und emotionale Intelligenz fördert. Die Wichtigkeit von Personas werden im Buch Designing for the Digital Age \citep[chapter 11]{goodwin2011designing} wie folgt beschrieben: \glqq A persona encapsulates and explains the most critical behavioral data in a way that designers and stakeholders can understand, remember, and relate to\grqq. Besonders in einem Umfeld wo... (erklären wieso empathie kleiner) \citep{hudson2009empathy}.
\newline{}
Es gibt verschiedene Möglichkeiten die Beschreibung und Charakterisierung einer Persona fest zu halten. Im Rahmen dieser Arbeit wird die Methode welche im Artikel \citep{interactions.acm2013online} \glqq User stories don't helpt users: introducing persona stories \grqq beschrieben ist. Die Beschreibung ist in eine Vorder- und Rückseite aufgeteilt. Die Vorderseite soll den Hintergrund und die Motivation einer Persona beschreiben und damit erklären, warum ein gewisse Merkmal auf diese Art und Weise implementiert werden soll. Die Rückseite beschreibt praktisch Details welche helfen Personas von diesem Typ zu erkennen. Diese Informationen werden auch beim rekrutieren von Testpersonen benötigt.

\subsubsection{User Stories vs Persona Stories}
- Seit wann gibt es User Stories, Wie sind sie entstanden
\newline{}
- Negative Aspekte: (Zitieren aus dem Artikel)
	1. Role ambiguity
	2. Sie erzählen was für eine Art Aktivität der Benutzer machen will aber nicht wie und wenn. Dabei wird auch die Frequenz ausser acht gelassen.
	3. User Stories sind in der Ich Person geschrieben und verhindern dadurch ein teil des Kreativen Prozesses.

\subsection{Mara Hürlimann}

\begin{table}[h!]
  \centering
  \begin{tabular}{ | c | m{5cm} | }
    \hline
    Photo & Steckbrief \\ \hline
    \begin{minipage}{.3\textwidth}
      \includegraphics[width=\linewidth, height=60mm]{images/batman.jpg}
    \end{minipage}
    &
    %\begin{minipage}[t]{5cm}
      \begin{itemize}
        \item Mara Hürlimann
        \item 49 Jährig
        \item aus Zürich (Stadt)
        \item arbeite als Lektorin
      \end{itemize}
    %\end{minipage}
    \\ \hline
  \end{tabular}
  \caption{Persona Steckbrief für Mara Hürlimann}\label{tbl:steckbriefmara}
\end{table}

\subsubsection{Hintergrund \& Motivation}
Mara nutzt die Dienstleistungen der KEP-Branche nicht regelmässig, was Sie aber nicht zu einer atypischen Nutzerin macht. Mara ist alleinerziehende Mutter und arbeite als Lektorin von zuhause aus. In ihrer Freizeit engagiert sie sich im Quartiersverein, organisiert kleinere Veranstaltungen für Flüchtlinge oder liest in der Sonne auf dem Balkon. Mara hat ein starkes Umweltbewusstsein und setzt für die Fortbewegung in der Stadt voll und ganz auf das Fahrrad. Für grössere Be- bzw. Entsorgungen mietet Mara ein Fahrzeug bei Mobility. Maras Umweltbewusst sein wird auch bei den wöchentlichen Einkäufen wiedergegeben. Sie kauft bewusst sasional und regional ein und versucht dadurch ihren CO2 Abdruck möglichst klein zu halten.

Maras älteste Tochter studiert Medialekünste in Berlin und hat bei ihrem letzten Besuch in Zürich ihre Unterlagen und ein Raspberry Pi, welches Sie für die Abschlusspräsentation benötigt, vergessen. Die Präsentation findet am Folgetag um 16 Uhr statt und hat keine Möglichkeit ein neues Gerät aufzutreiben. Mara will die Unterlagen inklusive dem Raspberry Pi mit einer Same Day Lieferung an ihre Tochter schicken und dabei ihre Umweltsphilosophie nicht verraten.

\subsubsection{Charaktermerkmal}
\begin{table}[]
\centering

\label{my-label}
\begin{tabular}{|l|l|}
\hline
Alter                                   & 20 - 60 Jahre        \\ \hline
Wohnort                                 & Stadtzentrum         \\ \hline
Ausbildung                              & Matura oder besser \\ \hline
Nutzung einer KEP Dienstleistung / Jahr & 1 - 2 mal            \\ \hline
\end{tabular}
\caption{Charaktermerkmal für Persona Mara Hürlimann}
\end{table}

Die Personen brauchen eine Express Kurier Dienstleistung nur für Notfälle oder sehr spezielle Ausnahmen. Sie haben keine Erfahrungen mit anderen KEP Dienstleistern und verlassen sich auf das im Markt etablierte Image eines Anbieters. Ihnen ist Transparenz und Zuverlässigkeit weicht. Sie wissen nicht wie der Prozess abläuft oder welche Informationen benötigt werden und sind beim Einkauf einer Dienstleistung auf Hilfe angewiesen.



\subsection{Peter}

\begin{table}[h!]
  \centering
  \begin{tabular}{ | c | m{5cm} | }
    \hline
    Photo & Steckbrief \\ \hline
    \begin{minipage}{.3\textwidth}
      \includegraphics[width=\linewidth, height=60mm]{images/batman.jpg}
    \end{minipage}
    &
    %\begin{minipage}[t]{5cm}
      \begin{itemize}
        \item Peter Elsener
        \item 28 Jährig
        \item aus Zürich (Stadt)
        \item arbeite als Projektleiter
      \end{itemize}
    %\end{minipage}
    \\ \hline
  \end{tabular}
  \caption{Persona Steckbrief für Peter Elsener}\label{tbl:steckbriefpeter}
\end{table}

\subsubsection{Hintergrund \& Motivation}
Peter arbeitet in einer kreativen Agentur als Projektleiter. Peter betreibt in seiner Freizeit viel Sport und spielt einmal in der Woche mit seinen Arbeitskollegen über den Mittag Fussball. Peter hat seine Prinzipien welche auch der Umwelt zugute kommen aber kann unter Umständen sehr opportun sein. Effizientes Arbeiten ist ihm sehr wichtig und das Gegenteil wird weder bei Mitarbeitern noch bei Dienstleistern akzeptiert. Peter verwaltet sehr viele Kunden und ist sehr darum Bemüht all deren Bedürfnisse zu 100\% Zufriedenheit zu erfüllen.

Trotz des digitalen Zeitalters involvieren viele Projekte von Peter immer noch Drucksachen. Diese Drucksachen werden vor der Massenproduktion dem Kunden ausgedruckt vorgelegt damit dieser ein "Gut zum Druck" geben kann. Üblicherweise werden diese Drucksachen erst kurz vor der Deadline fertig. Damit der Kunde schnellstmöglich das Okay geben kann, werden diese Ausdrucke per Express Kurier verschickt. Viele Kunden von Peter haben ihre Räumlichkeiten ausserkantonal und sind deshalb nicht über einen lokalen Fahrradkurierdienst erreichbar.

Peter benötigt dafür eine Same Day Lieferung und will eine verlässlichen und transparenten Express Kurier. Pflichtbewusst wie Peter ist, hat er den Express Kurier bei seinen Kunden schon budgetiert und will nicht mit unvorhergesehen Kosten überrascht werden.

\subsubsection{Charaktermerkmal}
\begin{table}[]
\centering

\label{my-label}
\begin{tabular}{|l|l|}
\hline
Alter                                   & 20 - 40 Jahre        \\ \hline
Wohnort                                 & Stadt oder Land         \\ \hline
Ausbildung                              & Bachelor oder besser \\ \hline
Nutzung einer KEP Dienstleistung / Jahr & 50 - 100 mal            \\ \hline
\end{tabular}
\caption{Charaktermerkmal für Persona Peter Elsener}
\end{table}

Die Personen benötigen regelmässig Express Kurierdienstleistungen und können gut mit Webbasierten Plattformen umgehen. Sie wissen welche Informationen für eine Erfolgreiche Lieferung benötigt werden und brauchen bei der Eingabe keinen Assistenten. Sie wollen alle sich wiederholen Tasks z. B. \glqq Abholadresse eingeben\grqq, da sie fast immer die gleiche ist, automatisieren bzw. speichern können.

\subsection{Laura Energie}
\begin{table}[h!]
  \centering
  \begin{tabular}{ | c | m{5cm} | }
    \hline
    Photo & Steckbrief \\ \hline
    \begin{minipage}{.3\textwidth}
      \includegraphics[width=\linewidth, height=60mm]{images/batman.jpg}
    \end{minipage}
    &
    %\begin{minipage}[t]{5cm}
      \begin{itemize}
        \item Laura Energie
        \item 37 Jährig
        \item aus Aarau
        \item Inhaberin eines Verlags
      \end{itemize}
    %\end{minipage}
    \\ \hline
  \end{tabular}
  \caption{Persona Steckbrief für Laura Energie}\label{tbl:steckbrieflaura}
\end{table}

\subsubsection{Hintergrund \& Motivation}
Laura ist Geschäftsführerin eines Verlages und publiziert ein monatliches Stadt/Land Magazin, welches sich stark mit den Themen Nachhaltigkeit, Familie und Ernährung auseinander setzt. Laura ist ein starke Persönlichkeit und trennt ihr Arbeits- und Privatleben strikt. Ihr Geschäft wird rein wirtschaftlich betrieben und was keinen Gewinn abwirft, wird nicht verfolgt.

Trotz der Mehrkosten die ein nachhaltiger KEP Dienstleister mit sich bringt, wittert Laura die grosse Marketing Idee. Sie hat einen sehr guten Geschäftssinn und schon früh bemerkt dass mit Grün und Bio Labels sehr einfach zu werben ist. Als Geschäftsfrau ist ihr die Umwelt nicht gerade an erster Stelle aber auch sie will wenn möglich etwas zum Umweltschutz bei tragen.


Lauras Verlag macht Wöchentlich Aboabschlüsse im 2-Stelligen Bereich und will die Ausgabe mit einem Expresskurier verschicken. Laura ist mehr an einer API Anbindung an ihre Abonomentsverwaltungssoftware intressiert, als an einer Maske wo die neuen Adressen eingetragen werden. Für eine begrenzte Zeit wäre ein Workaround möglich. Sie interessiert sich sowieso viel mehr für die Zahlen, welche eine solche Webapplikation liefern könnte.

\subsubsection{Charaktermerkmal}
\begin{table}[]
\centering

\label{my-label}
\begin{tabular}{|l|l|}
\hline
Alter                                   & 30 - 50 Jahre        \\ \hline
Wohnort                                 & Stadt oder Land         \\ \hline
Ausbildung                              & N/A  \\ \hline
Nutzung einer KEP Dienstleistung / Jahr & > 100 mal            \\ \hline
\end{tabular}
\caption{Charaktermerkmal für Persona Laura Energie}
\end{table}

Die Personen sind meist Inhaber oder verantwortlich für die Bereiche Logistik und/oder Versand ihres Unternehmen. Sie brauchen die KEP Dienstleistung als ein Teil ihres Businessprozesses und sind an einer total automatisierung intressiert. Ihr Interesse an einer Webapplikation liegt alleine im Auswerten von Zahlen und Statistiken.



\section{Persona Stories}
\label{sec:personastories}

 Anforderungen werden mithilfe von Persona Stories in der Form \glqq <Persona[:Rolle]> macht <Aufgabe>[, um <Ziel> zu erreichen]\grqq festgehalten. Die Platzhalter in den eckigen Klammern sind optional und sollen ausgefüllt werden, wenn dabei der Sinn der Story klarer wird. Zusätzlich werden Akzeptanzkriterien für die Stories definiert, welche in einem späteren Schritt als Akzeptanztests dienen. Die Priorität wird mit dem Kano Model bestimmt, welches im Kapitel \ref{sec:priorisierung} beschrieben und auf die folgenden Persona Stories angewendet worden ist. Die Priorität wird nur der Schönheits halber auch in der Persona Storie Karte aufgeführt.


\subsection{Mara}

\addcontentsline{toc}{subsection}{SC-001 Abholadresse eingeben}
\begin{usecase}
\addtitle{SC-001}{Abholadresse eingeben}
\addscenario{Story}{
	\item Mara gibt eine Abholadresse ein.
}
\addscenario{Akzeptanzkriterien}{
	\item Mara sieht während der Eingabe Vorschläge von Adressen welche im Versorgungsgebiet liegen.
	\item Mara sieht nach Abschluss der Eingabe ob die Adresse im Versorgungsgebiet liegt.
	\item Mara wird nach der Eingabe zum nächsten Eingabeschritt weitergeleitet.
}
\addfield{Priorität}{Hoch}

\end{usecase}

\addcontentsline{toc}{subsection}{SC-002 Lieferadresse eingeben}
\begin{usecase}
\addtitle{SC-002}{Lieferadresse eingeben}
\addscenario{Story}{
	\item Mara gibt eine Lieferadresse ein.
}
\addscenario{Akzeptanzkriterien}{
	\item Mara sieht während der Eingabe Vorschläge von Adressen welche im Versorgungsgebiet liegen.
	\item Mara sieht nach Abschluss der Eingabe ob die Adresse im Versorgungsgebiet liegt.
	\item Mara wird nach der Eingabe zum nächsten Eingabeschritt weitergeleitet.
}
\addfield{Priorität}{Hoch}

\end{usecase}

\newpage{}

\addcontentsline{toc}{subsection}{SC-003 Pakete Dimensionen eingeben}
\begin{usecase}
\addtitle{SC-003}{Pakete Dimensionen eingeben}
\addscenario{Story}{
	\item Mara gibt die Pakete Dimensionen (Gewicht, Länge, Breite und Grösse) ein.
}
\addscenario{Akzeptanzkriterien}{
	\item Mara sieht ob die Dimension versendet werden können.
	\item Mara wird nach der Eingabe zum nächsten Eingabeschritt weitergeleitet.
}
\addfield{Priorität}{Hoch}

\end{usecase}

\addcontentsline{toc}{subsection}{SC-004 Abholuhrzeit eingeben}
\begin{usecase}
\addtitle{SC-004}{Abholuhrzeit eingeben}
\addscenario{Story}{
	\item Mara gibt eine Abholurzeit ein um die verfügbaren Routen und deren Preis (Begriffs definition) zu sehen.
}
\addscenario{Akzeptanzkriterien}{
	\item Mara kann eine verfügbare Route auswählen.
	\item Mara wird nach der Auswahl zum nächsten Eingabeschritt weitergeleitet.
}
\addfield{Priorität}{Hoch}

\end{usecase}

\newpage{}

\addcontentsline{toc}{subsection}{SC-005 Kontaktinformationen eingeben}
\begin{usecase}
\addtitle{SC-005}{Kontaktinformationen eingeben}
\addscenario{Story}{
	\item Mara gibt Kontaktinformation (Name, Telefonnummer, evt. Stockwerk) für den Abhol sowie den Zielort ein.
}
\addscenario{Akzeptanzkriterien}{
	\item Mara wird nach der Eingabe zum nächsten Eingabeschritt weitergeleitet.
}
\addfield{Priorität}{Hoch}

\end{usecase}

\addcontentsline{toc}{subsection}{SC-006 Zollinformationen eingeben}
\begin{usecase}
\addtitle{SC-006}{Zollinformationen eingeben}
\addscenario{Story}{
	\item Mara gibt bei grenzüberschreitenden Paketen zusätzliche Informationen für den reibungslosen Ablauf beim Zoll ein
}
\addscenario{Akzeptanzkriterien}{
	\item Mara sieht allfällige Dokumente welche ausgedruckt und mitgeschickt werden müssen
	\item Mara wird nach der Eingabe zum nächsten Eingabeschritt weitergeleitet.
}
\addfield{Priorität}{Hoch}

\end{usecase}

\newpage{}

\addcontentsline{toc}{subsection}{SC-007 Dienstleistung kaufen}
\begin{usecase}
\addtitle{SC-007}{Dienstleistung kaufen}
\addscenario{Story}{
	\item Mara kauft die Dienstleistung um den Auftrag abzuschliessen.
}
\addscenario{Akzeptanzkriterien}{
	\item Mara bekommt eine Bestätigung (Mail/Drucken) für ihren Auftragsabschluss
	\item Mara bekommt eine Nummer mit welcher der Lieferprozess verfolgt werden kann.
	\item
}
\addfield{Priorität}{Mittel}

\end{usecase}

\addcontentsline{toc}{subsection}{SC-008 Lieferung verfolgen}
\begin{usecase}
\addtitle{SC-008}{Lieferung verfolgen}
\addscenario{Story}{
	\item Mara gibt die Trackingnummer ein um den Lieferung Status zu überprüfen.
}
\addscenario{Akzeptanzkriterien}{
	\item Mara sieht den Fortschritt ihrer Lieferung
}
\addfield{Priorität}{Niedrig}

\end{usecase}


\newpage{}


\addcontentsline{toc}{subsection}{SC-009 Unterstützung Anfordern}
\begin{usecase}
\addtitle{SC-009}{Unterstützung Anfordern}
\addscenario{Story}{
	\item Mara will mit einer Person telefonieren um Hilfe beim Kauf der Dienstleistung zu bekommen.
}
\addscenario{Akzeptanzkriterien}{
	\item Mara sieht die Telefonnummer oder Emailadresse
	\item Mara sieht die Auftragsnummer
}
\addfield{Priorität}{Niedrig}

\end{usecase}

\newpage{}
\subsection{Peter}

\addcontentsline{toc}{subsection}{SC-010 Versandangaben Eingeben}
\begin{usecase}
\addtitle{SC-010}{Versandangaben Eingeben}
\addscenario{Story}{
	\item Peter gibt alle Informationen für den Versand ein um die verfügbaren Routen und deren Preis (Begriffsdefinition) zu sehen.
}
\addscenario{Akzeptanzkriterien}{
	\item Peter kann eine verfügbare Route auswählen
	\item Peter wird zum Zoll Informationen Schritt weitergeleitet
}
\addfield{Priorität}{Hoch}

\end{usecase}



% Das gleich wie SC-006
\addcontentsline{toc}{subsection}{SC-011 Zollinformationen Eingeben}
\begin{usecase}
\addtitle{SC-011}{Zollinformationen Eingeben}
\addscenario{Story}{
	\item Peter gibt alle relevanten Informationen für den Zoll ein.
}
\addscenario{Akzeptanzkriterien}{
	\item Peter sieht allfällige Dokumente welche ausgedruckt und mitgeschickt werden müssen
	\item Peter wird nach der Eingabe zum Bezahlung Schritt weitergeleitet.
}
\addfield{Priorität}{Hoch}

\end{usecase}

\newpage{}

\addcontentsline{toc}{subsection}{SC-012 Abholadressen Speichern}
\begin{usecase}
\addtitle{SC-012}{Abholadressen Speichern}
\addscenario{Story}{
	\item Peter speichert die Abholadresse und Kontaktinformationen seines Büros um sie für den späteren Gebrauch wiederverwenden zu können.
}
\addscenario{Akzeptanzkriterien}{
	\item Peter sieht seine gespeicherten Adressen und kann sie bearbeiten
	\item Peter kann die gespeicherte Adresse beim SC-009 auswählen
}
\addfield{Priorität}{Mittel}

\end{usecase}


\addcontentsline{toc}{subsection}{SC-013 Zieladressen Speichern}
\begin{usecase}
\addtitle{SC-013}{Zieladressen Speichern}
\addscenario{Story}{
	\item Peter speichert die Zieladressen und Kontaktinformationen seiner Kunden um sie für den späteren Gebrauch wiederverwenden zu können.
}
\addscenario{Akzeptanzkriterien}{
	\item Peter sieht seine gespeicherten Adressen und kann sie bearbeiten
	\item Peter kann die gespeicherte Adresse beim SC-009 auswählen
}
\addfield{Priorität}{Mittel}

\end{usecase}

\newpage{}


\addcontentsline{toc}{subsection}{SC-014 Auftrag abbrechen}
\begin{usecase}
\addtitle{SC-014}{Auftrag abbrechen}
\addscenario{Story}{
	\item Peter bricht einen noch nicht gestarteten Auftrag ab.
}
\addscenario{Akzeptanzkriterien}{
	\item Peter sieht dass der Auftrag abgebrochen ist.
	\item Peter bekommt eine Bestätigung dass der Auftrag nicht ausgeführt wird.
}
\addfield{Priorität}{Mittel}

\end{usecase}


\addcontentsline{toc}{subsection}{SC-015 Aufträge auflisten}
\begin{usecase}
\addtitle{SC-015}{Aufträge auflisten}
\addscenario{Story}{
	\item Peter liste die im System erfassten Aufträge.
}
\addscenario{Akzeptanzkriterien}{
	\item Peter sieht alle bereits erfassten Aufträge.
	\item Peter kann einzelne Aufträge anklicken und sieht deren Details.
}
\addfield{Priorität}{Niedrig}

\end{usecase}

\newpage{}

\subsection{Laura}

\addcontentsline{toc}{subsection}{SC-016 One-To-Many Auftrag erfassen}
\begin{usecase}
\addtitle{SC-016}{One-To-Many Auftrag erfassen}
\addscenario{Story}{
	\item Laura möchte einen Auftrag erfassen welcher von einer Abholadresse an mehrere Lieferadressen versendet wird
}
\addscenario{Akzeptanzkriterien}{
	\item Laura sieht den Auftrag und die dafür berechnete Route.
	\item Laura sieht wann der Auftrag gestartet wird und wann er beendet sein soll.
}
\addfield{Priorität}{Niedrig}

\end{usecase}

\addcontentsline{toc}{subsection}{SC-017 Auftragsstatus überprüfen}
\begin{usecase}
\addtitle{SC-017}{Auftragsstatus überprüfen}
\addscenario{Story}{
	\item Laura kann den Status ihrer Aufträge überprüfen.
}
\addscenario{Akzeptanzkriterien}{
	\item Laura sieht den Status ihrer abgeschlossenen und aktuellen Aufträge.
}
\addfield{Priorität}{Niedrig}

\end{usecase}

\addcontentsline{toc}{subsection}{SC-018 Routen auswählen}
\begin{usecase}
\addtitle{SC-018}{Routen auswählen}
\addscenario{Story}{
	\item Laura kann für ihren Auftrag zwischen verschiedenen Routen auswählen
}
\addscenario{Akzeptanzkriterien}{
	\item Laura sieht Routen welche sich im Preis und/oder der Dauer unterscheiden können.
}
\addfield{Priorität}{Niedrig}

\end{usecase}


\newpage{}


\section{Priorisierung}
\label{sec:priorisierung}

\subsection{Kano Modell}
Das Kano Modell welches nach seinem Erfinder Noriaki Kano, Professor an der Universität in Tokio, benannt ist, bestimmt die Notwendigkeit von Kundenwünschen und Erwartungen (Zitieren). Beim Kano Modell wird zwischen fünf Qualitätsebenen unterschieden.

\begin{description}
	\item[Basis-Merkmale] sind für den Benutzer so selbstverständlich, dass ihm/ihr ihre Notwendigkeit erst beim nicht vorhanden sein auffällt.
	\item[Leistungs-Merkmal] werden vom Benutzer bewusst wahr genommen und sorgen für Zufriedenheit oder beseitigen Unzufriedenheit.
	\item[Begeisterungs-Merkmale] überraschen den Benutzer und bringen ihm mehr Nutzen und Funktionalität.
	\item[Unerhebliche Merkmale] sind dem Benutzer im Falle des Vorhandensein sowie auch Fehlens egal.
	\item[Rückweisungs-Merkmale] machen den Benutzer unzufrieden, beim nicht vorhanden sein jedoch nicht zufrieden.
\end{description}

Diese Qualitätsebenen geben die Priorität bei der Entwicklung der Kundenwünsche vor. Basis-Merkmale haben die Priorität Hoch, Leistungs-Merkmale haben die Priorität Mittel und Begeisterungs-Merkmale bekommen die Priorität Niedrig. Die restlichen 2 Merkmale können bei der Priorisierung vernachlässigt werden da Sie zum einen nicht implementiert werden sollten und zum andern nur bei fehlen der anderen 3 Merkmale relevant werden.

Der Benutzer antwortet bezüglich eines Produktwunsches bzw. einer Anforderung auf eine positiv formuliert und eine negativ formulierte Frage:
\begin{itemize}
	\item Funktional: Was würden Sie sagen, wenn die Applikation ..... macht.
	\item Dysfunktional: Was würden Sie sagen, wenn die Applikation ..... NICHT macht.
\end{itemize}

Dabei stehen ihm folgende Antworten zur Auswahl:

\begin{itemize}
	\item Das würde mich sehr freuen
	\item Das setzte ich voraus
	\item Das ist mir egal
	\item Das nehme ich gerade noch hin
	\item Das würde mich sehr stören
\end{itemize}

Aus der Kombinationen der Antwort auf die Positive und der Antwort auf die Negative Frage, kann der Wunsch bzw. die Anforderung in eine der oben genannten 5 Qualitätsebenen eingeteilt werden. Es bestehen folgende Möglichkeiten:

\begin{table}[h!]
\centering
\label{tbl:kanoantworten}
\begin{tabular}{clclc}
\multicolumn{2}{c}{\textbf{Funktionale Antwort}} & \multicolumn{2}{c}{\textbf{Dysfunktionale Antwort}} & \textbf{Merkmal}      \\
Das setze ich voraus                & +          & Das würde mich stören                 & =           & Basis-Merkmal         \\
Das würde mich sehr freuen          & +          & Das würde mich sehr stören            & =           & Leistungs-Merkmal     \\
Das würde mich sehr freuen          & +          & Das ist mir egal                      & =           & Begeisterungs-Merkmal \\
Das ist mir egal                    & +          & Das ist mir egal                      & =           & Unerhebliches Merkmal \\
Das würde mich sehr stören          & +          & Das setze ich voraus                  & =           & Rückweisungs-Merkmal
\end{tabular}
\caption{Antwort Möglichkeiten beim Kano Modell}
\end{table}

Antworten welche in der Tabelle \ref{tbl:kanoantworten} nicht aufgelistet sind, sind unlogisch und werden für die Bewertung ignoriert.

\subsection{Durchführung???}
Die im Kapitel \ref{sec:personastories} aufgeführten Persona Story Karten werden in der folgenden Tabelle nach dem Kano Model priorisiert. Der Besseren Übersicht halber werden die möglichen Antworten auf die funktionale sowie die dysfunktionale Frage mit Grossbuchstaben wie folgt dargestellt.
\begin{enumerate*}[label={\Alph*)},font={\color{red!50!black}\bfseries}]
	\item Das würde mich sehr freuen
	\item Das setzte ich voraus
	\item Das ist mir egal
	\item Das nehme ich gerade noch hin
	\item Das würde mich sehr stören
\end{enumerate*}. Die Persona Stories werde nur durch ihre Nummer SC-XXX identifiziert.

\begin{table}[h!]
\centering
\label{my-label}
\begin{tabular}{cccccc}
Persona Story & \begin{tabular}[c]{@{}c@{}}Funktionale\\ Antwort\end{tabular} &   & \begin{tabular}[c]{@{}c@{}}Dysfunktionale\\ Antwort\end{tabular} &   & Ergebnis              \\
SC-001        & B                                                             & + & E                                                                & = & Basis-Merkmal         \\
SC-002        & B                                                             & + & E                                                                & = & Basis-Merkmal         \\
SC-003        & B                                                             & + & E                                                                & = & Basis-Merkmal         \\
SC-004        & B                                                             & + & E                                                                & = & Basis-Merkmal         \\
SC-005        & B                                                             & + & E                                                                & = & Basis-Merkmal         \\
SC-006        & B                                                             & + & E                                                                & = & Basis-Merkmal         \\
SC-007        & A                                                             & + & E                                                                & = & Leistungs-Merkmal     \\
SC-008        & A                                                             & + & C                                                                & = & Begeisterungs-Merkmal \\
SC-009        & A                                                             & + & C                                                                & = & Begeisterungs-Merkmal \\
SC-010        & B                                                             & + & E                                                                & = & Basis-Merkmal         \\
SC-011        & B                                                             & + & E                                                                & = & Basis-Merkmal         \\
SC-012        & A                                                             & + & E                                                                & = & Leistungs-Merkmal     \\
SC-013        & A                                                             & + & E                                                                & = & Leistungs-Merkmal     \\
SC-014        & A                                                             & + & E                                                                & = & Leistungs-Merkmal     \\
SC-015        & A                                                             & + & C                                                                & = & Begeisterungs-Merkmal \\
SC-016        & A                                                             & + & C                                                                & = & Begeisterungs-Merkmal \\
SC-017        & A                                                             & + & C                                                                & = & Begeisterungs-Merkmal \\
SC-018        & A                                                             & + & C                                                                & = & Begeisterungs-Merkmal
\end{tabular}
\caption{My caption}
\end{table}










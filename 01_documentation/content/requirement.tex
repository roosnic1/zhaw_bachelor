%!TEX root = ../doc.tex
\chapter{Anforerdungsanalyse}
\label{sec:anforderungsanalyse}


\section{Einleitung}
Mit dem User Centered Design Prozess konzentriert sich die Anforderungsanalyse sehr stark auf den Benutzer. Dennoch wurden mit den wichtigsten Stakeholders Interviews geführt, in welchen sich zeigte was für Aufgaben und die Webapplikation für die Firma lösen soll.




\section{Stakeholders}
\begin{itemize}
	\item David Emmerth
	\item Nick Blake
	\item Jürgen (Lobo)
\end{itemize}

\subsection{Zusammenfassung der Interviews}

\section{Personas}

\subsection{Einleitung}
Personas sind Urbilder welche die verschiedenen Ziele und Verhaltensmuster der typischen Benutzer beschreiben sollen. Sie unterscheiden sich von anderen Methoden hauptsächlich durch ihre Geschichte erzählenden Art, welche bei Entwicklern und Designern besonders die sozialen und emotionale Intelligenz fördert. Die Wichtigkeit von Personas werden im Buch Designing for the Digital Age \citep[chapter 11]{goodwin2011designing} wie folgt beschrieben: \glqq A persona encapsulates and explains the most critical behavioral data in a way that designers and stakeholders can understand, remember, and relate to\grqq. Besonders in einem Umfeld wo... (erklären wieso empathie kleiner) \citep{hudson2009empathy}.
\newline{}
Es gibt verschiedene Möglichkeiten die Beschreibung und Charakterisierung einer Persona fest zu halten. Im Rahmen dieser Arbeit wird die Methode welche im Artikel \citep{interactions.acm2013online} \glqq User stories don't helpt users: introducing persona stories \grqq beschrieben ist. Die Beschreibung ist in eine Vorder- und Rückseite aufgeteilt. Die Vorderseite soll den Hintergrund und die Motivation einer Persona beschreiben und damit erklären, warum ein gewisse Merkmal auf diese Art und Weise implementiert werden soll. Die Rückseite beschreibt praktisch Details welche helfen Personas von diesem Typ zu erkennen. Diese Informationen werden auch beim rekrutieren von Testpersonen benötigt.

\subsubsection{User Stories vs Persona Stories}
- Seit wann gibt es User Stories, Wie sind sie entstanden
\newline{}
- Negative Aspekte: (Zitieren aus dem Artikel)
	1. Role ambiguity
	2. Sie erzählen was für eine Art Aktivität der Benutzer machen will aber nicht wie und wenn. Dabei wird auch die Frequenz ausser acht gelassen.
	3. User Stories sind in der Ich Person geschrieben und verhindern dadurch ein teil des Kreativen Prozesses.

\subsection{Mara Hürlimann}

\begin{table}[h!]
  \centering
  \begin{tabular}{ | c | m{5cm} | }
    \hline
    Photo & Steckbrief \\ \hline
    \begin{minipage}{.3\textwidth}
      \includegraphics[width=\linewidth, height=60mm]{images/batman.jpg}
    \end{minipage}
    &
    %\begin{minipage}[t]{5cm}
      \begin{itemize}
        \item Mara Hürlimann
        \item 49 Jährig
        \item aus Zürich (Stadt)
        \item arbeite als Lektorin
      \end{itemize}
    %\end{minipage}
    \\ \hline
  \end{tabular}
  \caption{Persona Steckbrief für Mara Hürlimann}\label{tbl:steckbriefmara}
\end{table}

\subsubsection{Hintergrund \& Motivation}
Mara nutzt die Dienstleistungen der KEP-Branche nicht regelmässig, was Sie aber nicht zu einer atypischen Nutzerin macht. Mara ist alleinerziehende Mutter und arbeite als Lektorin von zuhause aus. In ihrer Freizeit engagiert sie sich im Quartiersverein, organisiert kleinere Veranstaltungen für Flüchtlinge oder liest in der Sonne auf dem Balkon. Mara hat ein starkes Umweltbewusstsein und setzt für die Fortbewegung in der Stadt voll und ganz auf das Fahrrad. Für grössere Be- bzw. Entsorgungen mietet Mara ein Fahrzeug bei Mobility. Maras Umweltbewusst sein wird auch bei den wöchentlichen Einkäufen wiedergegeben. Sie kauft bewusst sasional und regional ein und versucht dadurch ihren CO2 Abdruck möglichst klein zu halten.

Maras älteste Tochter studiert Medialekünste in Berlin und hat bei ihrem letzten Besuch in Zürich ihre Unterlagen und ein Raspberry Pi, welches Sie für die Abschlusspräsentation benötigt, vergessen. Die Präsentation findet am Folgetag um 16 Uhr statt und hat keine Möglichkeit ein neues Gerät aufzutreiben. Mara will die Unterlagen inklusive dem Raspberry Pi mit einer Same Day Lieferung an ihre Tochter schicken und dabei ihre Umweltsphilosophie nicht verraten.

\subsubsection{Charaktermerkmal}
\begin{table}[]
\centering

\label{my-label}
\begin{tabular}{|l|l|}
\hline
Alter                                   & 20 - 60 Jahre        \\ \hline
Wohnort                                 & Stadtzentrum         \\ \hline
Ausbildung                              & Matura oder besser \\ \hline
Nutzung einer KEP Dienstleistung / Jahr & 1 - 2 mal            \\ \hline
\end{tabular}
\caption{Charaktermerkmal für Persona Mara Hürlimann}
\end{table}

Die Personen brauchen eine Express Kurier Dienstleistung nur für Notfälle oder sehr spezielle Ausnahmen. Sie haben keine Erfahrungen mit anderen KEP Dienstleistern und verlassen sich auf das im Markt etablierte Image eines Anbieters. Ihnen ist Transparenz und Zuverlässigkeit weicht. Sie wissen nicht wie der Prozess abläuft oder welche Informationen benötigt werden und sind beim Einkauf einer Dienstleistung auf Hilfe angewiesen.



\subsection{Peter}

\begin{table}[h!]
  \centering
  \begin{tabular}{ | c | m{5cm} | }
    \hline
    Photo & Steckbrief \\ \hline
    \begin{minipage}{.3\textwidth}
      \includegraphics[width=\linewidth, height=60mm]{images/batman.jpg}
    \end{minipage}
    &
    %\begin{minipage}[t]{5cm}
      \begin{itemize}
        \item Peter Elsener
        \item 28 Jährig
        \item aus Zürich (Stadt)
        \item arbeite als Projektleiter
      \end{itemize}
    %\end{minipage}
    \\ \hline
  \end{tabular}
  \caption{Persona Steckbrief für Peter Elsener}\label{tbl:steckbriefpeter}
\end{table}

\subsubsection{Hintergrund \& Motivation}
Peter arbeitet in einer kreativen Agentur als Projektleiter. Peter betreibt in seiner Freizeit viel Sport und spielt einmal in der Woche mit seinen Arbeitskollegen über den Mittag Fussball. Peter hat seine Prinzipien welche auch der Umwelt zugute kommen aber kann unter Umständen sehr opportun sein. Effizientes Arbeiten ist ihm sehr wichtig und das Gegenteil wird weder bei Mitarbeitern noch bei Dienstleistern akzeptiert. Peter verwaltet sehr viele Kunden und ist sehr darum Bemüht all deren Bedürfnisse zu 100\% Zufriedenheit zu erfüllen.

Trotz des digitalen Zeitalters involvieren viele Projekte von Peter immer noch Drucksachen. Diese Drucksachen werden vor der Massenproduktion dem Kunden ausgedruckt vorgelegt damit dieser ein "Gut zum Druck" geben kann. Üblicherweise werden diese Drucksachen erst kurz vor der Deadline fertig. Damit der Kunde schnellstmöglich das Okay geben kann, werden diese Ausdrucke per Express Kurier verschickt. Viele Kunden von Peter haben ihre Räumlichkeiten ausserkantonal und sind deshalb nicht über einen lokalen Fahrradkurierdienst erreichbar.

Peter benötigt dafür eine Same Day Lieferung und will eine verlässlichen und transparenten Express Kurier. Pflichtbewusst wie Peter ist, hat er den Express Kurier bei seinen Kunden schon budgetiert und will nicht mit unvorhergesehen Kosten überrascht werden.

\subsubsection{Charaktermerkmal}
\begin{table}[]
\centering

\label{my-label}
\begin{tabular}{|l|l|}
\hline
Alter                                   & 20 - 40 Jahre        \\ \hline
Wohnort                                 & Stadt oder Land         \\ \hline
Ausbildung                              & Bachelor oder besser \\ \hline
Nutzung einer KEP Dienstleistung / Jahr & 50 - 100 mal            \\ \hline
\end{tabular}
\caption{Charaktermerkmal für Persona Peter Elsener}
\end{table}

Die Personen benötigen regelmässig Express Kurierdienstleistungen und können gut mit Webbasierten Plattformen umgehen. Sie wissen welche Informationen für eine Erfolgreiche Lieferung benötigt werden und brauchen bei der Eingabe keinen Assistenten. Sie wollen alle sich wiederholen Tasks z. B. \glqq Abholadresse eingeben\grqq, da sie fast immer die gleiche ist, automatisieren bzw. speichern können.

\subsection{Laura Energie}
\begin{table}[h!]
  \centering
  \begin{tabular}{ | c | m{5cm} | }
    \hline
    Photo & Steckbrief \\ \hline
    \begin{minipage}{.3\textwidth}
      \includegraphics[width=\linewidth, height=60mm]{images/batman.jpg}
    \end{minipage}
    &
    %\begin{minipage}[t]{5cm}
      \begin{itemize}
        \item Laura Energie
        \item 37 Jährig
        \item aus Aarau
        \item Inhaberin eines Verlags
      \end{itemize}
    %\end{minipage}
    \\ \hline
  \end{tabular}
  \caption{Persona Steckbrief für Laura Energie}\label{tbl:steckbrieflaura}
\end{table}

\subsubsection{Hintergrund \& Motivation}
Laura ist Geschäftsführerin eines Verlages und publiziert ein monatliches Stadt/Land Magazin, welches sich stark mit den Themen Nachhaltigkeit, Familie und Ernährung auseinander setzt. Laura ist ein starke Persönlichkeit und trennt ihr Arbeits- und Privatleben strikt. Ihr Geschäft wird rein wirtschaftlich betrieben und was keinen Gewinn abwirft, wird nicht verfolgt.

Trotz der Mehrkosten die ein nachhaltiger KEP Dienstleister mit sich bringt, wittert Laura die grosse Marketing Idee. Sie hat einen sehr guten Geschäftssinn und schon früh bemerkt dass mit Grün und Bio Labels sehr einfach zu werben ist. Als Geschäftsfrau ist ihr die Umwelt nicht gerade an erster Stelle aber auch sie will wenn möglich etwas zum Umweltschutz bei tragen.


Lauras Verlag macht Wöchentlich Aboabschlüsse im 2-Stelligen Bereich und will die Ausgabe mit einem Expresskurier verschicken. Laura ist mehr an einer API Anbindung an ihre Abonomentsverwaltungssoftware intressiert, als an einer Maske wo die neuen Adressen eingetragen werden. Für eine begrenzte Zeit wäre ein Workaround möglich. Sie interessiert sich sowieso viel mehr für die Zahlen, welche eine solche Webapplikation liefern könnte.

\subsubsection{Charaktermerkmal}
\begin{table}[]
\centering

\label{my-label}
\begin{tabular}{|l|l|}
\hline
Alter                                   & 30 - 50 Jahre        \\ \hline
Wohnort                                 & Stadt oder Land         \\ \hline
Ausbildung                              & N/A  \\ \hline
Nutzung einer KEP Dienstleistung / Jahr & > 100 mal            \\ \hline
\end{tabular}
\caption{Charaktermerkmal für Persona Laura Energie}
\end{table}

Die Personen sind meist Inhaber oder verantwortlich für die Bereiche Logistik und/oder Versand ihres Unternehmen. Sie brauchen die KEP Dienstleistung als ein Teil ihres Businessprozesses und sind an einer total automatisierung intressiert. Ihr Interesse an einer Webapplikation liegt alleine im Auswerten von Zahlen und Statistiken.



\section{Persona Stories}

 Anforderungen werden mithilfe von Persona Stories in der Form \glqq <Persona[:Rolle]> macht <Aufgabe>[, um <Ziel> zu erreichen]\grqq festgehalten. Die Platzhalter in den eckigen Klammern sind optional und sollen ausgefüllt werden, wenn dabei der Sinn der Story klarer wird.


\subsection{Mara}

\addcontentsline{toc}{subsection}{SC-001 Abholadresse eingeben}
\begin{usecase}
\addtitle{SC-001}{Abholadresse eingeben}
\addscenario{Story}{
	\item Mara gibt eine Abholadresse ein.
}
\addscenario{Akzeptanzkriterien}{
	\item Mara sieht während der Eingabe Vorschläge von Adressen welche im Versorgungsgebiet liegen.
	\item Mara sieht nach Abschluss der Eingabe ob die Adresse im Versorgungsgebiet liegt.
	\item Mara wird nach der Eingabe zum nächsten Eingabeschritt weitergeleitet.
}
\addfield{Priorität}{Hoch}
\addfield{Domain}{???}
\end{usecase}

\addcontentsline{toc}{subsection}{SC-002 Lieferadresse eingeben}
\begin{usecase}
\addtitle{SC-002}{Lieferadresse eingeben}
\addscenario{Story}{
	\item Mara gibt eine Lieferadresse ein.
}
\addscenario{Akzeptanzkriterien}{
	\item Mara sieht während der Eingabe Vorschläge von Adressen welche im Versorgungsgebiet liegen.
	\item Mara sieht nach Abschluss der Eingabe ob die Adresse im Versorgungsgebiet liegt.
	\item Mara wird nach der Eingabe zum nächsten Eingabeschritt weitergeleitet.
}
\addfield{Priorität}{Hoch}
\addfield{Domain}{???}
\end{usecase}

\addcontentsline{toc}{subsection}{SC-003 Pakete Dimensionen eingeben}
\begin{usecase}
\addtitle{SC-003}{Pakete Dimensionen eingeben}
\addscenario{Story}{
	\item Mara gibt die Pakete Dimensionen (Gewicht, Länge, Breite und Grösse) ein.
}
\addscenario{Akzeptanzkriterien}{
	\item Mara sieht ob die Dimension versendet werden können.
	\item Mara wird nach der Eingabe zum nächsten Eingabeschritt weitergeleitet.
}
\addfield{Priorität}{Hoch}
\addfield{Domain}{???}
\end{usecase}

\addcontentsline{toc}{subsection}{SC-004 Abholuhrzeit eingeben}
\begin{usecase}
\addtitle{SC-004}{Abholuhrzeit eingeben}
\addscenario{Story}{
	\item Mara gibt eine Abholurzeit ein um die verfügbaren Routen und deren Preis (Begriffs definition) zu sehen.
}
\addscenario{Akzeptanzkriterien}{
	\item Mara kann eine verfügbare Route auswählen.
	\item Mara wird nach der Auswahl zum nächsten Eingabeschritt weitergeleitet.
}
\addfield{Priorität}{Hoch}
\addfield{Domain}{???}
\end{usecase}

\addcontentsline{toc}{subsection}{SC-005 Kontaktinformationen eingeben}
\begin{usecase}
\addtitle{SC-005}{Kontaktinformationen eingeben}
\addscenario{Story}{
	\item Mara gibt Kontaktinformation (Name, Telefonnummer, evt. Stockwerk) für den Abhol sowie den Zielort ein.
}
\addscenario{Akzeptanzkriterien}{
	\item Mara wird nach der Eingabe zum nächsten Eingabeschritt weitergeleitet.
}
\addfield{Priorität}{Hoch}
\addfield{Domain}{???}
\end{usecase}

\addcontentsline{toc}{subsection}{SC-006 Zollinformationen eingeben}
\begin{usecase}
\addtitle{SC-006}{Zollinformationen eingeben}
\addscenario{Story}{
	\item Mara gibt bei grenzüberschreitenden Paketen zusätzliche Informationen für den reibungslosen Ablauf beim Zoll ein
}
\addscenario{Akzeptanzkriterien}{
	\item Mara sieht allfällige Dokumente welche ausgedruckt und mitgeschickt werden müssen
	\item Mara wird nach der Eingabe zum nächsten Eingabeschritt weitergeleitet.
}
\addfield{Priorität}{Hoch}
\addfield{Domain}{???}
\end{usecase}

\addcontentsline{toc}{subsection}{SC-007 Dienstleistung kaufen}
\begin{usecase}
\addtitle{SC-007}{Dienstleistung kaufen}
\addscenario{Story}{
	\item Mara kauft die Dienstleistung um den Auftrag abzuschliessen.
}
\addscenario{Akzeptanzkriterien}{
	\item Mara bekommt eine Bestätigung (Mail/Drucken) für ihren Auftragsabschluss
	\item Mara bekommt eine Nummer mit welcher der Lieferprozess verfolgt werden kann.
	\item
}
\addfield{Priorität}{Hoch}
\addfield{Domain}{???}
\end{usecase}

\addcontentsline{toc}{subsection}{SC-008 Lieferung verfolgen}
\begin{usecase}
\addtitle{SC-008}{Lieferung verfolgen}
\addscenario{Story}{
	\item Mara gibt die Trackingnummer ein um den Lieferung Status zu überprüfen.
}
\addscenario{Akzeptanzkriterien}{
	\item Mara sieht den Fortschritt ihrer Lieferung
}
\addfield{Priorität}{Hoch}
\addfield{Domain}{???}
\end{usecase}

\newpage{}
\subsection{Peter}

\addcontentsline{toc}{subsection}{SC-009 Versandangaben Eingeben}
\begin{usecase}
\addtitle{SC-009}{Versandangaben Eingeben}
\addscenario{Story}{
	\item Peter gibt alle Informationen für den Versand ein um die verfügbaren Routen und deren Preis (Begriffsdefinition) zu sehen.
}
\addscenario{Akzeptanzkriterien}{
	\item Peter kann eine verfügbare Route auswählen
	\item Peter wird zum Zoll Informationen Schritt weitergeleitet
}
\addfield{Priorität}{Hoch}
\addfield{Domain}{???}
\end{usecase}

% Das gleich wie SC-006
\addcontentsline{toc}{subsection}{SC-010 Zollinformationen Eingeben}
\begin{usecase}
\addtitle{SC-010}{Zollinformationen Eingeben}
\addscenario{Story}{
	\item Peter gibt alle relevanten Informationen für den Zoll ein.
}
\addscenario{Akzeptanzkriterien}{
	\item Peter sieht allfällige Dokumente welche ausgedruckt und mitgeschickt werden müssen
	\item Peter wird nach der Eingabe zum Bezahlung Schritt weitergeleitet.
}
\addfield{Priorität}{Hoch}
\addfield{Domain}{???}
\end{usecase}

\addcontentsline{toc}{subsection}{SC-011 Abholadressen Speichern}
\begin{usecase}
\addtitle{SC-011}{Abholadressen Speichern}
\addscenario{Story}{
	\item Peter speichert die Abholadresse und Kontaktinformationen seines Büros um sie für den späteren Gebrauch wiederverwenden zu können.
}
\addscenario{Akzeptanzkriterien}{
	\item Peter sieht seine gespeicherten Adressen und kann sie bearbeiten
	\item Peter kann die gespeicherte Adresse beim SC-009 auswählen
}
\addfield{Priorität}{Hoch}
\addfield{Domain}{???}
\end{usecase}

\addcontentsline{toc}{subsection}{SC-012 Zieladressen Speichern}
\begin{usecase}
\addtitle{SC-012}{Zieladressen Speichern}
\addscenario{Story}{
	\item Peter speichert die Zieladressen und Kontaktinformationen seiner Kunden um sie für den späteren Gebrauch wiederverwenden zu können.
}
\addscenario{Akzeptanzkriterien}{
	\item Peter sieht seine gespeicherten Adressen und kann sie bearbeiten
	\item Peter kann die gespeicherte Adresse beim SC-009 auswählen
}
\addfield{Priorität}{Hoch}
\addfield{Domain}{???}
\end{usecase}


\newpage{}
\subsection{Laura}


\section{Akzeptanztests}

\begin{description}
\item[US-1] End User sieht nach Eingabe des Start- und Zielort alle verfügbaren Routen.
\end{description}


\section{Priorisierung}
Die User Stories werden mit dem MoSCoW Prinzip priorisiert.
???




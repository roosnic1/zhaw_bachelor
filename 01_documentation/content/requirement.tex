%!TEX root = ../doc.tex
\chapter{Anforerdungsanalyse}
\label{sec:anforderungsanalyse}


\section{Einleitung}
Mit dem User Centered Design Prozess konzentriert sich die Anforderungsanalyse sehr stark auf den Benutzer. Die Interviews zeigten was für Aufgaben die Webapplikation für die Firma lösen soll.


\section{Stakeholders}
\begin{description}
	\item[David Emmerth] Radkurier und Social Entrepreneurship-Experte, Community Organiser für die Schweiz und Deutschland.
	\item[Nick Blake] Gründer von Imagine Cargo.
\end{description}

\section{Personas}
\label{sec:personas}

\subsection{Einleitung}
Personas sind Urbilder, welche die verschiedenen Ziele und Verhaltensmuster der typischen Benutzer beschreiben sollen. Sie unterscheiden sich von anderen Methoden hauptsächlich durch ihre geschichteerzählende Art, welche bei Entwicklern und Designern besonders die soziale und emotionale Intelligenz fördert. Die Wichtigkeit von Personas wird im Buch Designing for the Digital Age \citep[chapter 11]{goodwin2011designing} wie folgt beschrieben: \glqq A persona encapsulates and explains the most critical behavioral data in a way that designers and stakeholders can understand, remember, and relate to\grqq.
%Besonders in einem Umfeld wo... (erklären wieso empathie kleiner) \citep{hudson2009empathy}.
\newline{}
Es gibt verschiedene Möglichkeiten, die Beschreibung und Charakterisierung einer Persona festzuhalten. Im Rahmen dieser Arbeit wird die Methode angewendet, welche im Artikel \citep{interactions.acm2013online} \glqq User stories don't helpt users: introducing persona stories \grqq beschrieben ist. Die Beschreibung ist in eine Vorder- und Rückseite aufgeteilt. Die Vorderseite soll den Hintergrund und die Motivation einer Persona beschreiben und damit erklären, warum ein gewisses Merkmal auf diese Art und Weise implementiert werden soll. Die Rückseite beschreibt praktische Details, welche helfen, Personas von diesem Typ zu erkennen. Diese Informationen werden auch beim rekrutieren von Testpersonen benötigt.

\subsection{Mara Hürlimann}

\begin{table}[h!]
  \centering
  \begin{tabular}{ | c | m{5cm} | }
    \hline
    Photo & Steckbrief \\ \hline
    \begin{minipage}{.3\textwidth}
      \includegraphics[width=\linewidth, height=60mm]{images/batman.jpg}
    \end{minipage}
    &
    %\begin{minipage}[t]{5cm}
      \begin{itemize}
        \item Mara Hürlimann
        \item 49-Jährig
        \item aus Zürich (Stadt)
        \item arbeitet als Lektorin
      \end{itemize}
    %\end{minipage}
    \\ \hline
  \end{tabular}
  \caption{Persona Steckbrief für Mara Hürlimann}\label{tbl:steckbriefmara}
\end{table}

\subsubsection{Hintergrund \& Motivation}
Mara nutzt die Dienstleistungen der KEP-Branche nicht regelmässig, was sie aber nicht zu einer atypischen Nutzerin macht. Mara ist alleinerziehende Mutter und arbeitet als Lektorin von zuhause aus. In ihrer Freizeit engagiert sie sich im Quartierverein, organisiert kleinere Veranstaltungen für Flüchtlinge oder liest in der Sonne auf dem Balkon. Mara hat ein starkes Umweltbewusstsein und setzt für die Fortbewegung in der Stadt voll und ganz auf das Fahrrad. Für grössere Be- bzw. Entsorgungen mietet Mara ein Fahrzeug bei Mobility. Maras Umweltbewusstsein zeigt sich auch bei den wöchentlichen Einkäufen. Sie kauft bewusst saisonal und regional ein und versucht dadurch ihren, CO2 Abdruck möglichst klein zu halten.

Maras älteste Tochter studiert Mediale Künste in Berlin und hat Bei ihrem letzten Besuch in Zürich ihre Unterlagen und ein Raspberry Pi, welches sie für die Abschlusspräsentation benötigt, vergessen. Die Präsentation findet am Folgetag um 16 Uhr statt und die junge Frau hat keine Möglichkeit, ein neues Gerät aufzutreiben. Mara will die Unterlagen inklusive dem Raspberry Pi mit einer Same-Day-Lieferung an ihre Tochter schicken und dabei ihre Umweltsphilosophie nicht verraten.

\subsubsection{Charaktermerkmal}
\begin{table}[]
\centering

\label{my-label}
\begin{tabular}{|l|l|}
\hline
Alter                                   & 20 - 60 Jahre        \\ \hline
Wohnort                                 & Stadtzentrum         \\ \hline
Ausbildung                              & Matura oder besser \\ \hline
Nutzung einer KEP-Dienstleistung / Jahr & 1 - 2 mal            \\ \hline
\end{tabular}
\caption{Charaktermerkmal für Persona Mara Hürlimann}
\end{table}

Die Personen brauchen eine Expresskurier-Dienstleistung nur für Notfälle oder spezielle Ausnahmen. Sie haben keine Erfahrungen mit anderen KEP-Dienstleistern und verlassen sich auf das im Markt etablierte Image eines Anbieters. Ihnen ist Transparenz und Zuverlässigkeit wichtig. Sie wissen nicht, wie der Prozess abläuft oder welche Informationen benötigt werden und sind beim Einkauf einer Dienstleistung auf Hilfe angewiesen.



\subsection{Peter Elsener}

\begin{table}[h!]
  \centering
  \begin{tabular}{ | c | m{5cm} | }
    \hline
    Photo & Steckbrief \\ \hline
    \begin{minipage}{.3\textwidth}
      \includegraphics[width=\linewidth, height=60mm]{images/batman.jpg}
    \end{minipage}
    &
    %\begin{minipage}[t]{5cm}
      \begin{itemize}
        \item Peter Elsener
        \item 28-Jährig
        \item aus Zürich (Stadt)
        \item arbeitet als Projektleiter
      \end{itemize}
    %\end{minipage}
    \\ \hline
  \end{tabular}
  \caption{Persona Steckbrief für Peter Elsener}\label{tbl:steckbriefpeter}
\end{table}

\subsubsection{Hintergrund \& Motivation}
Peter arbeitet in einer kreativen Agentur als Projektleiter. Peter treibt in seiner Freizeit viel Sport und spielt einmal in der Woche mit seinen Arbeitskollegen über Mittag Fussball. Peter hat seine Prinzipien, welche auch der Umwelt zugute kommen, kann aber auch sehr opportun sein. Effizientes Arbeiten ist ihm sehr wichtig. Das Gegenteil akzeptiert er weder bei Mitarbeitern noch bei Dienstleistern. Peter verwaltet viele Kunden und ist sehr darum Bemüht all deren Bedürfnisse zur vollsten Zufriedenheit zu erfüllen.

Trotz des digitalen Zeitalters involvieren viele Projekte von Peter immer noch Drucksachen. Diese werden vor der Massenproduktion dem Kunden ausgedruckt vorgelegt, damit dieser ein \glqq{}Gut zum Druck\grqq geben kann. Üblicherweise werden diese Drucksachen erst kurz vor der Deadline fertig. Damit der Kunde schnellstmöglich das Okay geben kann, werden diese Ausdrucke per Expresskurier verschickt. Viele Kunden von Peter haben ihre Räumlichkeiten ausserkantonal und sind deshalb nicht über einen lokalen Fahrradkurierdienst erreichbar.

Peter benötigt dafür eine Same-Day-Lieferung und will einen verlässlichen und transparenten Expresskurier. Pflichtbewusst wie Peter ist, hat er den Expresskurier bei seinen Kunden schon budgetiert und will nicht mit unvorhergesehen Kosten überrascht werden.

\subsubsection{Charaktermerkmal}
\begin{table}[]
\centering

\label{my-label}
\begin{tabular}{|l|l|}
\hline
Alter                                   & 20 - 40 Jahre        \\ \hline
Wohnort                                 & Stadt oder Land         \\ \hline
Ausbildung                              & Bachelor oder besser \\ \hline
Nutzung einer KEP-Dienstleistung / Jahr & 50 - 100 mal            \\ \hline
\end{tabular}
\caption{Charaktermerkmal für Persona Peter Elsener}
\end{table}

Die Personen benötigen regelmässig Expresskurierdienstleistungen und können gut mit webbasierten Plattformen umgehen. Sie wissen, welche Informationen für eine erfolgreiche Lieferung benötigt werden und brauchen bei der Eingabe keinen Assistenten. Sie wollen alle sich wiederholenden Tasks z. B. \glqq Abholadresse eingeben\grqq automatisieren beziehungsweise speichern können.

\subsection{Laura Energie}
\begin{table}[h!]
  \centering
  \begin{tabular}{ | c | m{5cm} | }
    \hline
    Photo & Steckbrief \\ \hline
    \begin{minipage}{.3\textwidth}
      \includegraphics[width=\linewidth, height=60mm]{images/batman.jpg}
    \end{minipage}
    &
    %\begin{minipage}[t]{5cm}
      \begin{itemize}
        \item Laura Energie
        \item 37-Jährig
        \item aus Aarau
        \item Inhaberin eines Verlags
      \end{itemize}
    %\end{minipage}
    \\ \hline
  \end{tabular}
  \caption{Persona Steckbrief für Laura Energie}\label{tbl:steckbrieflaura}
\end{table}

\subsubsection{Hintergrund \& Motivation}
Laura ist Geschäftsführerin eines Verlages und publiziert ein monatliches Stadt/Land Magazin, welches sich stark mit den Themen Nachhaltigkeit, Familie und Ernährung auseinandersetzt. Laura ist eine starke Persönlichkeit und trennt ihr Arbeits- und Privatleben strikt. Ihr Geschäft wird rein wirtschaftlich betrieben, was keinen Gewinn abwirft, wird nicht weiterverfolgt.

Trotz der Mehrkosten die ein nachhaltiger KEP-Dienstleister mit sich bringt, wittert Laura die grosse Marketingstrategie. Sie hat einen sehr guten Geschäftssinn und schon früh bemerkt, dass mit Grün- und Biolabels sehr einfach zu werben ist. Als Geschäftsfrau ist ihr die Umwelt nicht gerade an erster Stelle, aber auch sie will etwas zum Umweltschutz bei tragen.


Lauras Verlag macht wöchentlich Aboabschlüsse im zweistelligen Bereich und will die Ausgabe mit einem Expresskurier verschicken. Laura ist mehr an einer API-Anbindung an ihre Abonnementsverwaltungssoftware interessiert, als an einer Maske, wo die neuen Adressen eingetragen werden. Für eine begrenzte Zeit wäre ein Workaround möglich. Sie interessiert sich sowieso viel mehr für die Zahlen. Eine solche Webapplikation könnte diese liefern.

\subsubsection{Charaktermerkmal}
\begin{table}[]
\centering

\label{my-label}
\begin{tabular}{|l|l|}
\hline
Alter                                   & 30 - 50 Jahre        \\ \hline
Wohnort                                 & Stadt oder Land         \\ \hline
Ausbildung                              & N/A  \\ \hline
Nutzung einer KEP-Dienstleistung / Jahr & > 100 mal            \\ \hline
\end{tabular}
\caption{Charaktermerkmal für Persona Laura Energie}
\end{table}

Die Personen sind meist Inhaber oder verantwortlich für die Bereiche Logistik und/oder Versand ihres Unternehmens. Sie brauchen die KEP-Dienstleistung als ein Teil ihres Businessprozesses und sind an einer totalen Automatisierung interessiert. Ihr Interesse an einer Webapplikation liegt alleine im Auswerten von Zahlen und Statistiken.


\newpage{}
\section{Persona Stories}
\label{sec:personastories}

 Anforderungen werden mithilfe von Persona Stories in der Form \glqq <Persona[:Rolle]> macht <Aufgabe>[, um <Ziel> zu erreichen]\grqq festgehalten. Die Platzhalter in den eckigen Klammern sind optional und sollen ausgefüllt werden, wenn dabei der Sinn der Story klarer wird. Zusätzlich werden Akzeptanzkriterien für die Stories definiert, welche in einem späteren Schritt als Akzeptanztests dienen. Die Priorität wird mit dem Kano Model bestimmt, welches im Kapitel \ref{sec:priorisierung} beschrieben und auf die folgenden Persona Stories angewendet worden ist. Die Priorität wird nur der Schönheits halber auch in der Persona Storie Karte aufgeführt.


\subsection{Mara Hürlimann}

\addcontentsline{toc}{subsection}{PS-001 Abholadresse eingeben}
\begin{usecase}
\addtitle{PS-001}{Abholadresse eingeben}
\addscenario{Story}{
	\item Mara gibt eine Abholadresse ein.
}
\addscenario{Akzeptanzkriterien}{
	\item Mara sieht während der Eingabe Vorschläge von Adressen, welche im Versorgungsgebiet liegen.
	\item Mara sieht nach Abschluss der Eingabe, ob die Adresse im Versorgungsgebiet liegt.
	\item Mara wird nach der Eingabe zum nächsten Eingabeschritt weitergeleitet.
}
\addfield{Priorität}{Hoch}

\end{usecase}

\addcontentsline{toc}{subsection}{PS-002 Lieferadresse eingeben}
\begin{usecase}
\addtitle{PS-002}{Lieferadresse eingeben}
\addscenario{Story}{
	\item Mara gibt eine Lieferadresse ein.
}
\addscenario{Akzeptanzkriterien}{
	\item Mara sieht während der Eingabe Vorschläge von Adressen, welche im Versorgungsgebiet liegen.
	\item Mara sieht nach Abschluss der Eingabe, ob die Adresse im Versorgungsgebiet liegt.
	\item Mara wird nach der Eingabe zum nächsten Eingabeschritt weitergeleitet.
}
\addfield{Priorität}{Hoch}

\end{usecase}

\newpage{}

\addcontentsline{toc}{subsection}{PS-003 Pakete Dimensionen eingeben}
\begin{usecase}
\addtitle{PS-003}{Pakete Dimensionen eingeben}
\addscenario{Story}{
	\item Mara gibt die Pakete Dimensionen (Gewicht, Länge, Breite und Grösse) ein.
}
\addscenario{Akzeptanzkriterien}{
	\item Mara sieht, ob die Dimension versendet werden können.
	\item Mara wird nach der Eingabe zum nächsten Eingabeschritt weitergeleitet.
}
\addfield{Priorität}{Hoch}

\end{usecase}

\addcontentsline{toc}{subsection}{PS-004 Abholuhrzeit eingeben}
\begin{usecase}
\addtitle{PS-004}{Abholuhrzeit eingeben}
\addscenario{Story}{
	\item Mara gibt eine Abholurzeit ein, um die verfügbaren Routen und deren Preis zu sehen.
}
\addscenario{Akzeptanzkriterien}{
	\item Mara kann eine verfügbare Route auswählen.
	\item Mara wird nach der Auswahl zum nächsten Eingabeschritt weitergeleitet.
}
\addfield{Priorität}{Hoch}

\end{usecase}

\newpage{}

\addcontentsline{toc}{subsection}{PS-005 Kontaktinformationen eingeben}
\begin{usecase}
\addtitle{PS-005}{Kontaktinformationen eingeben}
\addscenario{Story}{
	\item Mara gibt Kontaktinformation (Name, Telefonnummer, evt. Stockwerk) für den Abhol- sowie den Zielort ein.
}
\addscenario{Akzeptanzkriterien}{
	\item Mara wird nach der Eingabe zum nächsten Eingabeschritt weitergeleitet.
}
\addfield{Priorität}{Hoch}

\end{usecase}

\addcontentsline{toc}{subsection}{PS-006 Zollinformationen eingeben}
\begin{usecase}
\addtitle{PS-006}{Zollinformationen eingeben}
\addscenario{Story}{
	\item Mara gibt bei grenzüberschreitenden Paketen zusätzliche Informationen für den reibungslosen Ablauf beim Zoll ein.
}
\addscenario{Akzeptanzkriterien}{
	\item Mara sieht allfällige Dokumente, welche ausgedruckt und mitgeschickt werden müssen.
	\item Mara wird nach der Eingabe zum nächsten Eingabeschritt weitergeleitet.
}
\addfield{Priorität}{Hoch}

\end{usecase}

\newpage{}

\addcontentsline{toc}{subsection}{PS-007 Dienstleistung kaufen}
\begin{usecase}
\addtitle{PS-007}{Dienstleistung kaufen}
\addscenario{Story}{
	\item Mara kauft die Dienstleistung, um den Auftrag abzuschliessen.
}
\addscenario{Akzeptanzkriterien}{
	\item Mara bekommt eine Bestätigung (Mail/Drucken) für ihren Auftragsabschluss.
	\item Mara bekommt eine Nummer, mit welcher der Lieferprozess verfolgt werden kann.
}
\addfield{Priorität}{Mittel}

\end{usecase}

\addcontentsline{toc}{subsection}{PS-008 Lieferung verfolgen}
\begin{usecase}
\addtitle{PS-008}{Lieferung verfolgen}
\addscenario{Story}{
	\item Mara gibt die Trackingnummer ein, um den Status der Lieferung zu überprüfen.
}
\addscenario{Akzeptanzkriterien}{
	\item Mara sieht den Fortschritt ihrer Lieferung.
}
\addfield{Priorität}{Niedrig}

\end{usecase}


\newpage{}


\addcontentsline{toc}{subsection}{PS-009 Unterstützung anfordern}
\begin{usecase}
\addtitle{PS-009}{Unterstützung Anfordern}
\addscenario{Story}{
	\item Mara will mit einer Person telefonieren, um Hilfe beim Kauf der Dienstleistung zu bekommen.
}
\addscenario{Akzeptanzkriterien}{
	\item Mara sieht die Telefonnummer oder Emailadresse.
	\item Mara sieht die Auftragsnummer.
}
\addfield{Priorität}{Niedrig}

\end{usecase}

\newpage{}
\subsection{Peter Elsener}

\addcontentsline{toc}{subsection}{PS-010 Versandangaben eingeben}
\begin{usecase}
\addtitle{PS-010}{Versandangaben Eingeben}
\addscenario{Story}{
	\item Peter gibt alle Informationen für den Versand ein, um die verfügbaren Routen und deren Preis zu sehen.
}
\addscenario{Akzeptanzkriterien}{
	\item Peter kann eine verfügbare Route auswählen.
	\item Peter wird zum Schritt \glqq{}Zoll Informationen\grqq weitergeleitet.
}
\addfield{Priorität}{Hoch}

\end{usecase}



% Das gleich wie PS-006
\addcontentsline{toc}{subsection}{PS-011 Zollinformationen eingeben}
\begin{usecase}
\addtitle{PS-011}{Zollinformationen Eingeben}
\addscenario{Story}{
	\item Peter gibt alle relevanten Informationen für den Zoll ein.
}
\addscenario{Akzeptanzkriterien}{
	\item Peter sieht allfällige Dokumente, welche ausgedruckt und mitgeschickt werden müssen.
	\item Peter wird nach der Eingabe zum Schritt \glqq{}Bezahlung\grqq weitergeleitet.
}
\addfield{Priorität}{Hoch}

\end{usecase}

\newpage{}

\addcontentsline{toc}{subsection}{PS-012 Abholadressen speichern}
\begin{usecase}
\addtitle{PS-012}{Abholadressen Speichern}
\addscenario{Story}{
	\item Peter speichert die Abholadresse und Kontaktinformationen seines Büros, um sie für den späteren Gebrauch wiederverwenden zu können.
}
\addscenario{Akzeptanzkriterien}{
	\item Peter sieht seine gespeicherten Adressen und kann sie bearbeiten.
	\item Peter kann die gespeicherte Adresse beim PS-009 auswählen.
}
\addfield{Priorität}{Mittel}

\end{usecase}


\addcontentsline{toc}{subsection}{PS-013 Zieladressen speichern}
\begin{usecase}
\addtitle{PS-013}{Zieladressen Speichern}
\addscenario{Story}{
	\item Peter speichert die Zieladressen und Kontaktinformationen seiner Kunden, um sie für den späteren Gebrauch wiederverwenden zu können.
}
\addscenario{Akzeptanzkriterien}{
	\item Peter sieht seine gespeicherten Adressen und kann sie bearbeiten.
	\item Peter kann die gespeicherte Adresse beim PS-009 auswählen.
}
\addfield{Priorität}{Mittel}

\end{usecase}

\newpage{}


\addcontentsline{toc}{subsection}{PS-014 Auftrag abbrechen}
\begin{usecase}
\addtitle{PS-014}{Auftrag abbrechen}
\addscenario{Story}{
	\item Peter bricht einen noch nicht gestarteten Auftrag ab.
}
\addscenario{Akzeptanzkriterien}{
	\item Peter sieht, dass der Auftrag abgebrochen ist.
	\item Peter bekommt eine Bestätigung, dass der Auftrag nicht ausgeführt wird.
}
\addfield{Priorität}{Mittel}

\end{usecase}


\addcontentsline{toc}{subsection}{PS-015 Aufträge auflisten}
\begin{usecase}
\addtitle{PS-015}{Aufträge auflisten}
\addscenario{Story}{
	\item Peter listet die im System erfassten Aufträge auf.
}
\addscenario{Akzeptanzkriterien}{
	\item Peter sieht alle bereits erfassten Aufträge.
	\item Peter kann einzelne Aufträge anklicken und sieht deren Details.
}
\addfield{Priorität}{Niedrig}

\end{usecase}

\newpage{}

\subsection{Laura Energie}

\addcontentsline{toc}{subsection}{PS-016 One-To-Many Auftrag erfassen}
\begin{usecase}
\addtitle{PS-016}{One-To-Many Auftrag erfassen}
\addscenario{Story}{
	\item Laura möchte einen Auftrag erfassen, welcher von einer Abholadresse an mehrere Lieferadressen versendet wird.
}
\addscenario{Akzeptanzkriterien}{
	\item Laura sieht den Auftrag und die dafür berechnete Route.
	\item Laura sieht, wann der Auftrag gestartet wird und wann er beendet sein soll.
}
\addfield{Priorität}{Niedrig}

\end{usecase}

\addcontentsline{toc}{subsection}{PS-017 Auftragsstatus überprüfen}
\begin{usecase}
\addtitle{PS-017}{Auftragsstatus überprüfen}
\addscenario{Story}{
	\item Laura kann den Status ihrer Aufträge überprüfen.
}
\addscenario{Akzeptanzkriterien}{
	\item Laura sieht den Status ihrer abgeschlossenen und aktuellen Aufträge.
}
\addfield{Priorität}{Niedrig}

\end{usecase}

\addcontentsline{toc}{subsection}{PS-018 Routen auswählen}
\begin{usecase}
\addtitle{PS-018}{Routen auswählen}
\addscenario{Story}{
	\item Laura kann für ihren Auftrag zwischen verschiedenen Routen auswählen.
}
\addscenario{Akzeptanzkriterien}{
	\item Laura sieht Routen, welche sich im Preis und/oder der Dauer unterscheiden können.
}
\addfield{Priorität}{Niedrig}

\end{usecase}


\newpage{}


\section{Priorisierung}
\label{sec:priorisierung}

Die im Kapitel \ref{sec:personastories} aufgeführten Persona Story Karten werden in der folgenden Tabelle nach dem Kano Modell priorisiert. Das Kano Modell ist im Anhang \ref{sec:kanomodel} ausführlich beschrieben. Aufgrund der fehlenden Endbenutzern, wurde die Priorisierung der Anforderungen mit den beiden Stakeholdern durchgeführt. Der besseren Übersicht halber werden die möglichen Antworten auf die funktionale sowie die dysfunktionale Frage mit Grossbuchstaben wie folgt dargestellt.
\begin{enumerate*}[label={\Alph*)},font={\color{red!50!black}\bfseries}]
	\item Das würde mich sehr freuen
	\item Das setzt ich voraus
	\item Das ist mir egal
	\item Das nehme ich gerade noch hin
	\item Das würde mich sehr stören
\end{enumerate*}. Die Persona Stories werden nur durch ihre Nummer PS-XXX identifiziert.

\begin{table}[h!]
\centering
\label{my-label}
\begin{tabular}{cccccc}
Persona Story & \begin{tabular}[c]{@{}c@{}}Funktionale\\ Antwort\end{tabular} &   & \begin{tabular}[c]{@{}c@{}}Dysfunktionale\\ Antwort\end{tabular} &   & Ergebnis              \\
PS-001        & B                                                             & + & E                                                                & = & Basis-Merkmal         \\
PS-002        & B                                                             & + & E                                                                & = & Basis-Merkmal         \\
PS-003        & B                                                             & + & E                                                                & = & Basis-Merkmal         \\
PS-004        & B                                                             & + & E                                                                & = & Basis-Merkmal         \\
PS-005        & B                                                             & + & E                                                                & = & Basis-Merkmal         \\
PS-006        & B                                                             & + & E                                                                & = & Basis-Merkmal         \\
PS-007        & A                                                             & + & E                                                                & = & Leistungs-Merkmal     \\
PS-008        & A                                                             & + & C                                                                & = & Begeisterungs-Merkmal \\
PS-009        & A                                                             & + & C                                                                & = & Begeisterungs-Merkmal \\
PS-010        & B                                                             & + & E                                                                & = & Basis-Merkmal         \\
PS-011        & B                                                             & + & E                                                                & = & Basis-Merkmal         \\
PS-012        & A                                                             & + & E                                                                & = & Leistungs-Merkmal     \\
PS-013        & A                                                             & + & E                                                                & = & Leistungs-Merkmal     \\
PS-014        & A                                                             & + & E                                                                & = & Leistungs-Merkmal     \\
PS-015        & A                                                             & + & C                                                                & = & Begeisterungs-Merkmal \\
PS-016        & A                                                             & + & C                                                                & = & Begeisterungs-Merkmal \\
PS-017        & A                                                             & + & C                                                                & = & Begeisterungs-Merkmal \\
PS-018        & A                                                             & + & C                                                                & = & Begeisterungs-Merkmal
\end{tabular}
\caption{Priorisierung der Persona Stories nach Kano}
\end{table}

Im Anhang \ref{sec:kanomodel} sind die Basis-Merkmale beschrieben als Eingenschaften bzw. Funktionen, welche dem Benutzer auffallen, wenn sie nicht vorhanden sind. Im Rahmen dieser Bachelorthesis werden deshalb alle Basis-Merkmale im Prototypen umgesetzten.

\newpage{}
\section{Funktionale Anforderungen}
\label{sec:funktionaleanforderungen}

\addcontentsline{toc}{subsection}{FREQ-001 Adresse automatisch vervollständigen}
\begin{usecase}
\addtitle{FREQ-001}{Adresse automatisch vervollständigen}
\addfield{PS-Referenz}{PS-001,PS-002,PS-010}
\addfield{Beschreibung}{Die Applikation soll mit Hilfe einer geeigneten Bibliothek/API nach Eingabe von wenigen Buchstaben Vorschläge zur vollständigen Adresse machen.}
\end{usecase}

\addcontentsline{toc}{subsection}{FREQ-002 Adresse verifizieren}
\begin{usecase}
\addtitle{FREQ-002}{Adresse verifizieren}
\addfield{PS-Referenz}{PS-001,PS-002,PS-010}
\addfield{Beschreibung}{Die Applikation soll überprüfen, ob die eingegebene Adresse in einem der Versorgungsgebiete liegt, welche bewirtschaftet werden.}
\end{usecase}

\addcontentsline{toc}{subsection}{FREQ-003 Auftrag im Backend-System erstellen}
\begin{usecase}
\addtitle{FREQ-003}{Auftrag in Backend-System erstellen}
\addfield{PS-Referenz}{PS-001,PS-002,PS-004,PS-010}
\addfield{Beschreibung}{Die Applikation soll im Backend-System einen Auftrag mit einer Start- und Zieladresse sowie einer Start-Uhrzeit erstellen.}
\end{usecase}

\addcontentsline{toc}{subsection}{FREQ-004 Auftrag im Backend-System aktualisieren}
\begin{usecase}
\addtitle{FREQ-004}{Auftrag im Backend-System aktualisieren}
\addfield{PS-Referenz}{PS-003,PS-005,PS-006,PS-010,PS-011,PS-014}
\addfield{Beschreibung}{Die Applikation soll die zusätzlichen Angaben (z.B. Kontaktperson, Pakete Dimensionen, Gewicht) im Backend-System aktualisieren.}
\end{usecase}

\addcontentsline{toc}{subsection}{FREQ-005 Auftrag im Backend-System starten}
\begin{usecase}
\addtitle{FREQ-005}{Auftrag im Backend-System starten}
\addfield{PS-Referenz}{PS-007}
\addfield{Beschreibung}{Die Applikation soll den Auftrag im Backend-System starten.}
\end{usecase}

\addcontentsline{toc}{subsection}{FREQ-006 Auftragskosten in Rechnung stellen}
\begin{usecase}
\addtitle{FREQ-006}{Auftragskosten in Rechnung stellen}
\addfield{PS-Referenz}{PS-007}
\addfield{Beschreibung}{Die Applikation soll den Auftrag dem Benutzer in Rechnung stellen.}
\end{usecase}

\addcontentsline{toc}{subsection}{FREQ-007 Auftrag aus dem Backend-System laden}
\begin{usecase}
\addtitle{FREQ-007}{Auftrag aus dem Backend-System laden}
\addfield{PS-Referenz}{PS-008,PS-014,PS-015}
\addfield{Beschreibung}{Die Applikation soll Aufträge aus dem Backend-System laden.}
\end{usecase}

\addcontentsline{toc}{subsection}{FREQ-008 Auftragsstatus synchronisieren}
\begin{usecase}
\addtitle{FREQ-008}{Auftrag Status synchronisieren}
\addfield{PS-Referenz}{PS-009}
\addfield{Beschreibung}{Die Applikation soll den Auftragsstatus bei jeder Änderung synchronisieren.}
\end{usecase}

\addcontentsline{toc}{subsection}{FREQ-009 Benutzer registrieren}
\begin{usecase}
\addtitle{FREQ-009}{Benutzer registrieren}
\addfield{PS-Referenz}{PS-012,PS-013,PS-014,PS-015}
\addfield{Beschreibung}{Die Applikation soll einen neuen Benutzer persistent erstellen.}
\end{usecase}

\addcontentsline{toc}{subsection}{FREQ-010 Benutzer anmelden}
\begin{usecase}
\addtitle{FREQ-010}{Benutzer anmelden}
\addfield{PS-Referenz}{PS-012,PS-013,PS-014,PS-015}
\addfield{Beschreibung}{Die Applikation soll einen Benutzer mit dem korrekten Passwort anmelden.}
\end{usecase}

\addcontentsline{toc}{subsection}{FREQ-011 Benutzer verwalten}
\begin{usecase}
\addtitle{FREQ-011}{Benutzer verwalten}
\addfield{PS-Referenz}{PS-012,PS-013,PS-014,PS-015}
\addfield{Beschreibung}{Die Applikation soll die Benutzer verwalten.}
\end{usecase}

\addcontentsline{toc}{subsection}{FREQ-012 Daten speichern}
\begin{usecase}
\addtitle{FREQ-012}{Daten speichern}
\addfield{PS-Referenz}{PS-012,PS-013,PS-014,PS-015}
\addfield{Beschreibung}{Die Applikation soll Daten persistent speichern können.}
\end{usecase}


\addcontentsline{toc}{subsection}{FREQ-013 Daten laden}
\begin{usecase}
\addtitle{FREQ-013}{Daten laden}
\addfield{PS-Referenz}{PS-012,PS-013,PS-014,PS-015}
\addfield{Beschreibung}{Die Applikation soll abgespeicherte Daten laden können.}
\end{usecase}


\section{Nicht-Funktionale Anforderungen}
\label{sec:nichtfunktionaleanforderungen}


\addcontentsline{toc}{subsection}{NFREQ-001 Browser Kompatibel}
\begin{usecase}
\addtitle{NFREQ-001}{Browser Kompatibel}
\addfield{Beschreibung}{Die Applikation muss in einem Browser funktionieren.}
\end{usecase}

\addcontentsline{toc}{subsection}{NFREQ-002 LoBo Kompatibel}
\begin{usecase}
\addtitle{NFREQ-002}{LoBo Kompatibel}
\addfield{Beschreibung}{Die Applikation muss mit dem Backend-System LoBo funktionieren.}
\end{usecase}

\addcontentsline{toc}{subsection}{NFREQ-003 SSL Zertifikat}
\begin{usecase}
\addtitle{NFREQ-003}{SSL Zertifikat}
\addfield{Beschreibung}{Die Applikation soll über eine sichere HTTPS Verbindung mit dem Benutzer und dem Backend-System kommunizieren.}
\end{usecase}
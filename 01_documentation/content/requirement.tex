%!TEX root = ../doc.tex
\chapter{Anforerdungsanalyse}
\label{sec:anforderungsanalyse}


\section{Einleitung}
Die Anforderungen werden mithilfe von User Stories in der Form \glqq Als <Rolle> möchte ich <Ziel/Wunsch>, um <Nutzen>\grqq festgehalten.

\section{Rollen}
\subsection{End User (EU)}
\textcolor{darkgray}{
	Den End User detailiert beschreiben
}
\subsection{System Administrator (SA)}
\textcolor{darkgray}{
	System Admin of Lobo?
}

\section{User Stories}
\subsection{End User (EU)}

\begin{description}
\item[US-1] Als End User möchte ich Start- und Zielort eingeben, um die verfügbaren Routen zu sehen.
\end{description}

\section{Akzeptanztests}

\begin{description}
\item[US-1] End User sieht nach Eingabe des Start- und Zielort alle verfügbaren Routen.
\end{description}


\section{Priorisierung}
Die User Stories werden mit dem MoSCoW Prinzip priorisiert.
???




%!TEX root = ../doc.tex
\chapter{Einleitung}
\label{sec:Einleitung}

\section{Zusammenfassung}
\label{sec:zusammenfassung}
Web Applikationen werden heute von viele Menschen täglich genutzt. Emailverwaltung, Textbearbeitung, Routenplanung und Hotelreservierung sind einige Aufgaben, welche mit einer Applikation im Web erledigt werden. Die Benutzeroberflächen dieser Web Applikationen müssen einem grossen und diversem Publikum dienen. Dieses breite Anforderungsspektrum kann dazu führen, dass die Oberflächen zu kompliziert oder sogar unbrauchbar werden. Das entwickeln einer Web Anwendung mit einer positiven Benutzer Erfahrung benötigt deshalb eine strukturierte und umfassende Analyse des zu bewältigenden Prozesses und den zukünftigen Benutzern.

\section{Ausgangslage}
\label{sec:ausgangslage}
Das Startup ImagineCargo revolutioniert die Art und Weise, wie Pakete versendet werden. Anstelle von Lastwagen-Flugzeug-Lastwagen setzt das Unternehmen auf Fahrrad-Zug-Fahrrad. Die Mission: Gleicher  Preis, gleiche Geschwindigkeit aber 99\% Reduktion an CO2. Dazu ist es auf die neueste Technologie angewiesen. Für die Dispo und Backoffice Software Lobo sucht ImagineCargo nach einer Web Applikation die Eingabe von Abhohlort, Abhohlzeit und Lieferort für den Kunden ermöglicht. Alle möglichen Zugverbindungen je nach Abhohlzeit angibt. Automatisch die Lieferzeit je nach gewählter Abhohlzeit bestimmt und die Preise für alle möglichen Zugverbindungen anzeigt. Zusätzlich soll der CO2 Ausstosses je nach gewählter Verbindung berechnet werden.


\section{Ziel der Arbeit}
\label{sec:zielderarbeit}
Die bisherigen Prozesse und Strukturen des Unternehmens sollen kritisch betrachtet und analysiert werden. Es sollen konkrete Massnahmen zur Verbesserung konzipiert und umgesetzt werden. Mittels geeigneten Prozessen sollen die Anforderungen an eine Web Applikation von der Auftraggeberin und deren Endkunden erfasst werden. Es soll ein Konzept zur Optimierung der Benutzererfahrung im Web erstellt werden. Die vorgeschlagenen Massnahmen sollen mittels einem oder mehreren Prototypen strukturiert überprüft werden.

\section{Aufgabenstellung}
\label{sec:aufgabenstellung}
\begin{itemize}
  \item Recherche: Nachforschung über Velokurierer und das Express Kurier Geschäft im Allgemein anstellen. Herausfinden welches die üblichen Prozesse und Praktiken in der Branche sind. Zusätzliche Recherche über welche Software Produkte im Bereich Routenplanung bereits existieren.
  \item Ist-Analyse: Eine Analyse der vorhanden Prozesse sowie der bereits eingesetzten Software durchführen und die gewonnenen Ergebnisse dokumentieren.
  \item Anforderungsanalyse: Herausfinden welche Stakeholder existieren und mit persönlichen Interviews die Anforderungen aller Beteiligten aufnehmen. Alle Anforderungen an eine geeignete Web Applikation analysieren, priorisieren und strukturiert dokumentieren.
  \item Konzept: Anhand der Anforderungsanalyse ein Konzept für eine geeignete Web Applikation erstellen, welche das Ziel der Arbeit am besten unterstützt.
  \item Prototyp: Einen oder mehrere Prototypen zur Überprüfung des Konzepts implementieren.
  \item Testen: Umgesetztes Konzept mit den Stakeholdern auf ihre Richtigkeit und die tatsächliche Verbesserung des Prozesses testen und protokollieren. Prototypen werden mit Akzeptanztests auf ihre Fähigkeiten und Leistungen getestet.
\end{itemize}


\section{Erwartete Resultate}
\label{sec:erwarteteresultate}
\begin{itemize}
  \item Recherche: Recherchebericht mit den gewonnenen Erkenntnissen. Übersicht über bereits existierende Softwarelösungen.
  \item Ist-Analyse: Dokumentation der vorhanden Prozesse, deren Zusammenhänge und ihren Abhängigkeiten. Beschreibung der bereits genutzten Software.
  \item Anforderungsanalyse: Strukturierte Darstellung aller Anforderungen der Stakeholder.
  \item Konzept: Dokumentiertes Konzept mit Vorschlägen für Verbesserungen bzw. die Implementation einer Web Applikation.
  \item Prototyp: Einer oder mehrere präsentierbare Prototypen.
  \item Testen: Testprotokolle
\end{itemize}

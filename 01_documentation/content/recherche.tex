%!TEX root = ../doc.tex
\chapter{Recherche}
\label{sec:recherche}

\section{Kurier-, Express- und Paketdienstbranche}
\subsection{Schweiz}
Die Kurier-, Express- und Paketdienstbranche (kurz KEP Branche) ist ein altes und unscheinbares Gewerbe. In der Schweiz hingegen konnte sich der KEP Markt erst ab 1997 entwickeln, weil zuvor die schweizerische Post (Post AG) ein Monopol auf den gesamten Inland KEP Markt hatte. Das Postgesetz vom 30. April 1997\citep[]{postgesetz.sr783} öffnete diesen Markt für private Anbieter. Ab dem ersten 1. Januar 1998 war es per Gesetzt für private KEP Anbieter erlaubt Pakete mit einem Gewicht über 2 Kg und alle Expresssendungen zu transportieren. Am 1. Januar 2004 wurde das Gewichtslimit für Pakete auf 1 Kg gesenkt und per 1. April 2009 auf 50 Gram womit die schweizerische Post nur noch ein Teilmonopol auf den inländischen Briefmarkt hat. Seit der Öffnung des KEP Marktes haben sich in der Schweiz einige private KEP Dienstleisungsanbieter etabliert und unter dem Dachverband KEP\&Mail organisiert.
\begin{figure}[ht]
	\centering
  \includegraphics[width=0.66\textwidth]{images/kepmailMitglieder.png}
	\caption{Mitglieder von KEP\&Mail}
	\label{fig1:kepmail}
\end{figure}


\textcolor{darkgray}{
  Informationen über das Business.
  Zahlen zu verschickten Paketen und Distanzen (Aus der Pitchpresenation von ImagineCargo)
  Zahlen zu Kunden welche Aufträge über Web aufgeben. (Oder so ähnlich)
}


\section{Kurier und Logistik Software}
\textcolor{darkgray}{
  Grosse Tools vorstellen
  Vor und Nachteile hervorheben
  (Möglicherweise sind die Tools zu komplex???)
}


\section{Lobo}
\textcolor{darkgray}{
  Lobo und Jürgen vorstellen
  Kleine Übersicht der Software
  Wer sind die Kunden
}
%!TEX root = ../doc.tex
\pagenumbering{Roman}

\appendix
\chapter{Anhang}
\label{sec:Anhang}

\section{Projektmanagement}\label{projektmanagement}

\textcolor{darkgray}{
  \begin{itemize}
  \item Offizielle Aufgabenstellung, Projektauftrag
  \item (Zeitplan)
  \item (Besprechungsprotokolle oder Journals)
  \item CD mit dem vollständigen Bericht als pdf-File inklusive Film- und Fotomaterial
  \item (Schaltpläne und Ablaufschemata)
  \item (Spezifikationen u. Datenblätter der verwendeten Messgeräte und/oder Komponenten)
  \item (Berechnungen, Messwerte, Simulationsresultate)
  \item (Stoffdaten)
  \item (Fehlerrechnungen mit Messunsicherheiten)
  \item (Grafische Darstellungen, Fotos)
  \item (Datenträger mit weiteren Daten (z.B. Software-Komponenten) inkl. Verzeichnis der auf diesem Datenträger abgelegten Dateien)
  \item (Softwarecode)
  \end{itemize}
}

\newpage{}

\section{LoBo API}
\label{sec:loboAPI}
Im folgenden sind die möglichen Schnittstellen Endpunkte von LoBo beschrieben. Die Angaben stammen direkt aus der Beschreibung der Schnittstelle des Entwicklers und werden der Vollständigkeit halber aufgeführt.

\begin{description}
  \item[getGrantedActions] returns an array of actions covered by your license key
  \item[getProductList] Lists all publicly available products
  \item[getPaymentList] Lists all available payment methods
  \item[getPriceScale] Get details of the price scale for the given product and list all graduations.
  \item[createTask] Creates an internal (yet hidden) task. The returned tasktoken is a reference to access and change this task, eg. to add stops, set a new reference time, change the productid, ...
  \item[setProductId] Change the product of the given task. Next, you might want to call action='calculateTask' to query the updated cost and transit times.
  \item[setPaymentId] Change the payment for a given task (typically called just before action='orderTask')
  \item[setCustomerNumber] Change the customer of the given task. Please consider that the resulting costs will change if the newly assigned customer is associated with another discount system. Next, you might want to call action='calculateTask' to query the resulting cost.
  \item[setRefTime] Change the reference time of the given task. Next, you might want to call action='calculateTask' to query the updated transit times and/or if the the product is still available at the new time.
  \item[addStop] Verifies an address and, on success (statuscode > 0): adds the address as new stop to the given task
  \item[verifyAddress] Verifies an address. Returned placeid can be used
  \item[addStopByPlaceId] Add a place as stop based on the search with action='verifyAddress' or action='autoCompletePlaceAndStreet'
  \item[addStopByCustomerNumber] Add a customer as stop
  \item[deleteStop] Deletes the given stop from the given task
  \item[setStopSequence] Set a new sequence of the stops.
  \item[getStopList] Returns the current stop list including address information
  \item[calculateTask] Calculates the cost and transit time parameters.
  \item[optimizeTask] Makes a copy of the existing task and minimizes the cost by reordering the stops (traveling sales man problem).
  \item[setTaskInfo] Set additional plain text infos for given task. To reset a field, send empty value. All optional params (notepublic, noteinhouse, noteprivate, contactperson) that are not defined in a request remain unchanged.
  \item[setStopInfo] Set additional plain text infos for given stop. To reset a field, send empty value. All optional params (notepublic, noteinhouse, noteprivate, contactperson, placename) that are not defined in a request remain unchanged.
  \item[setStopTime] Set a fixed time for the given stop. The fixed time can be ser either relative to the task's reference time (same date), or on a different date, if appointeddate parameter is given. Optionally you can set the a timecondition 'at', 'as from', or 'by' the fixed time. To reset the fxied time of a stop, simply set the appointedtime paramter to 0.
  \item[orderTask] Place the order in the LoBo system. After this call no more changes can be made to task (via the API).
  \item[autoCompleteStreet] Autocomplete street name based on an combination of a right wildcard and similarity search. If a street spans accross an area of different zip codes and/or city names it will be found for every unique combination.
  \item[getCustomer] Get list of customers (matching filter criteria).
  \item[createCustomer] Create new customer.
  \item[getZoneCompilation] List zone details of a given (zone based) product. Zones can be either be based on geopolygons or on postal codes.
  \item[getStatistics] Get usage statistics.
\end{description}

\newpage{}
\section{Kano Modell}
\label{sec:kanomodel}
Das Kano Modell welches nach seinem Erfinder Noriaki Kano, Professor an der Universität in Tokio, benannt ist, bestimmt die Notwendigkeit von Kundenwünschen und Erwartungen (Zitieren). Beim Kano Modell wird zwischen fünf Qualitätsebenen unterschieden.

\begin{description}
  \item[Basis-Merkmale] sind für den Benutzer so selbstverständlich, dass ihm/ihr ihre Notwendigkeit erst beim nicht vorhanden sein auffällt.
  \item[Leistungs-Merkmal] werden vom Benutzer bewusst wahr genommen und sorgen für Zufriedenheit oder beseitigen Unzufriedenheit.
  \item[Begeisterungs-Merkmale] überraschen den Benutzer und bringen ihm mehr Nutzen und Funktionalität.
  \item[Unerhebliche Merkmale] sind dem Benutzer im Falle des Vorhandensein sowie auch Fehlens egal.
  \item[Rückweisungs-Merkmale] machen den Benutzer unzufrieden, beim nicht vorhanden sein jedoch nicht zufrieden.
\end{description}

Diese Qualitätsebenen geben die Priorität bei der Entwicklung der Kundenwünsche vor. Basis-Merkmale haben die Priorität Hoch, Leistungs-Merkmale haben die Priorität Mittel und Begeisterungs-Merkmale bekommen die Priorität Niedrig. Die restlichen 2 Merkmale können bei der Priorisierung vernachlässigt werden da Sie zum einen nicht implementiert werden sollten und zum andern nur bei fehlen der anderen 3 Merkmale relevant werden.

Der Benutzer antwortet bezüglich eines Produktwunsches bzw. einer Anforderung auf eine positiv formuliert und eine negativ formulierte Frage:
\begin{itemize}
  \item Funktional: Was würden Sie sagen, wenn die Applikation ..... macht.
  \item Dysfunktional: Was würden Sie sagen, wenn die Applikation ..... NICHT macht.
\end{itemize}

Dabei stehen ihm folgende Antworten zur Auswahl:

\begin{itemize}
  \item Das würde mich sehr freuen
  \item Das setzte ich voraus
  \item Das ist mir egal
  \item Das nehme ich gerade noch hin
  \item Das würde mich sehr stören
\end{itemize}

Aus der Kombinationen der Antwort auf die Positive und der Antwort auf die Negative Frage, kann der Wunsch bzw. die Anforderung in eine der oben genannten 5 Qualitätsebenen eingeteilt werden. Es bestehen folgende Möglichkeiten:

\begin{table}[h!]
\centering
\label{tbl:kanoantworten}
\begin{tabular}{clclc}
\multicolumn{2}{c}{\textbf{Funktionale Antwort}} & \multicolumn{2}{c}{\textbf{Dysfunktionale Antwort}} & \textbf{Merkmal}      \\
Das setze ich voraus                & +          & Das würde mich stören                 & =           & Basis-Merkmal         \\
Das würde mich sehr freuen          & +          & Das würde mich sehr stören            & =           & Leistungs-Merkmal     \\
Das würde mich sehr freuen          & +          & Das ist mir egal                      & =           & Begeisterungs-Merkmal \\
Das ist mir egal                    & +          & Das ist mir egal                      & =           & Unerhebliches Merkmal \\
Das würde mich sehr stören          & +          & Das setze ich voraus                  & =           & Rückweisungs-Merkmal
\end{tabular}
\caption{Antwort Möglichkeiten beim Kano Modell}
\end{table}

Antworten welche in der Tabelle \ref{tbl:kanoantworten} nicht aufgelistet sind, sind unlogisch und werden für die Bewertung ignoriert.

\newpage{}
\section{Interview Fragen}
\subsection{Stakeholder Interview Fragen}
\label{subsec:stakeholderfragen}
\subsubsection{Begriff Definitionen}
\begin{description}
  \item[Service/Product] damit ist die zu entwickelnde Webapplikation gemeint
  \item[Business] damit ist das Geschäft von Imagine Cargo gemeint
  \item[Projekt] damit ist die Entwicklung des Prototypen gemeint
\end{description}

\subsubsection{Generelle Fragen}
\begin{itemize}
  \item Was sollte der Service ihrer Meinung nach sein? (Können Sie den kompletten Service in eigenen Worten beschreiben?)
  \item Für wen soll der Service sein (Beispiele)?
  \begin{itemize}
    \item Für andere Kuriere?
    \item Für Endkunden? (In was für Branchen sind diese Kunden)
    \item B2B/B2C?
    \item Leads (potenzielle Kunden)?
  \end{itemize}
  \item Wann soll die erste Version des Services verfügbar sein?
  \item Was muss die erste Version beinhalten? (MVP)
  \item Welche Sorgen bereitet ihnen der Service?
  \item Was ist das Schlimmste was passieren kann?
  \item Was soll der Service für das Business erreichen?
  \begin{itemize}
    \item Gewinn
    \item Einsparungen
    \item Marke und Position im Markt beeinflussen
  \end{itemize}
  \item Wie definieren Sie persönlich Erfolg für diesen Service?
  \item Wo sehen Sie sich im Ablauf dieses Projektes?
  \item Wie sieht der Service 2 bzw. 5 Jahre nach einer erfolgreichen Implementierung aus?
\end{itemize}

\subsubsection{Spezifische Fragen}
\begin{itemize}
  \item Soll der Service dem Kunden Informationen anzeigen oder auch eine zu kaufende Dienstleistung anbieten?
  \item Welche Arbeitsschritte soll der Service ablösen?
  \item Welche Informationen soll der Kunden nach einem Einkauf erhalten? (Bestätigung, Tracking, Status)
  \item Welche Informationen wünschen Sie vom Kunden?
  \item Welche Informationen benötigen Sie minimal um erfolgreich einen Auftrag durchzuführen?
\end{itemize}

\subsubsection{Marketing/Sales Fragen}
\begin{itemize}
  \item Wer sind ihre Kunden und Benutzer heute und wie soll sich das in den nächsten 5 Jahren ändern?
  \item Wie passt der Service in die gesamte Servicestrategie?
  \item Welches sind die grössten Konkurrenten und was für Sorgen bereiten die ihnen?
  \item rscheidet sich das Produkt von der Konkurrenz?
  \item Welche 3/4 Qualitäten sollen Menschen mit dem Service und der Firma verbinden?
  \item Wieso benutzten Kunden diesen Service und nicht den eines Konkurrenten?
  \item Über was beschweren sich Kunden bzw. was wird am meisten verlangt? Und Wieso?
\end{itemize}

\subsubsection{Engineering Fragen}
\begin{itemize}
  \item Welche technologischen Entscheidungen wurden bereits getroffen? Wieso?
  \item Gibt es ein Diagramm welches das bereits existierende System beschreibt?
  \item Soll der Service nur eine Oberfläche für das Backend werden oder sollen auch andere Funktionen damit erledigt werden?
\end{itemize}


\subsubsection{Experten Fragen}
\begin{itemize}
  \item Was sind die typischen Demographien und Fähigkeiten der Benutzer und wie fest unterscheiden sich diese?
  \item Welche Unterschiede in den Benutzer Rollen und Aufgaben können erwartet werden?
  \item Welche Arbeitsabläufe und Praktiken können vor und nach der Benutzung erwartet  werden?
\end{itemize}


\subsection{Stakeholder Interview Zusammenfassung}
\label{subsec:interviewzusammen}
\subsubsection{Interview mit Nick Blake}
\subsubsection{Interview mit David Emmerth}
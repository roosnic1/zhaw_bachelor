%!TEX root = ../doc.tex
\chapter{Fazit}
\label{sec:fazit}
Trotz der klaren Strukturen welcher diese Bachelorarbeit zu Grunde lage, gab es ein einige Hürden zu überwinden beziehungs zu umgehen. Die Wahl diese verbesserung der Benutzererfahrung mit Hilfe des User Centered Design Prozesse zu bewältigen war total richtig. Die Ausführung jedoch benötigte bedeutend mehr Zeit als ursprünglich Angenommen und hätte ohne die Kompromisse gewaltig mehr Ressourcen gebraucht. Die korrekte Überprüfung der erstellten Prozesse, der Wireframes und des Prototypen mit Testbenutzern, welche keine Verbindung zum Projekt oder der Auftragsgeberfrima haben, wären sehr wichtig gewesen. Dank der iterativen Natur des User Centered Design Prozesses, ist dieses Versäumnis jedoch in der nächsten Iteration ohne Probleme nachzuholen. Im allgemeinen lässt dieses iterative Vorgehen mehr Fehler zu, was jedoch dazu führen kann mehr aus zu probieren und zu wagen. Das Buch von Kim Goodwin “Designing for the Digital Age” welches stark für die durchführung des User Centered Designs zu rate gezogen wurde, verweist bei der Suche nach Testbenutzern auf professionelle Firmen die solche Personen auftriben. Die naive Annahme in der freien Wirtschaft Menschen zu finden, welche helfen ein Produkt dass Sie selber brauchen zu verbessern, wurde klar widerlegt.
Die Wahl der Technologien führte trotz ihrer Neuheit zu keinen erheblichen Problemen. In gewissen Situationen war es schwierig Informationen oder Hilfe für kleiner Probleme beziehungsweise architektonische Fragen zu bekommen. Aber davon abgesehen waren sie die richtige Wahl um diesen Prototypen damit zu entwickeln. Imagine Cargo besitzt nun eine solide Grundlage um eine Web Applikation zu entwickeln, mit welcher ein grosser Teil der Benutzer eine angenehme Erfahrung bei der erstmaligen Benutzung des Services machen wird.


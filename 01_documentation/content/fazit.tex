%!TEX root = ../doc.tex
\chapter{Fazit}
\label{sec:fazit}
Trotz der klaren Strukturen, die dieser Bachelorarbeit zu Grunde lagen, gab es einige Hürden zu nehmen respektive zu umgehen. Die Wahl dieser Verbesserung der Benutzererfahrung mit Hilfe des User Centered Design Prozesse zu bewältigen war richtig. Die Ausführung jedoch benötigte bedeutend mehr Zeit als ursprünglich angenommen und hat ohne die Kompromisse gewaltig mehr Ressourcen gebraucht. Die korrekte Überprüfung der erstellten Prozesse, der Wireframes und des Prototypen durch Testbenutzern, die keine Verbindung zum Projekt oder der Auftraggeberfirma haben, wäre sehr wichtig gewesen. Dank der iterativen Natur des User Centerd Design Prozesses ist dieses Versäumnis jedoch in der nächsten Iteration ohne Probleme nachzuholen. Im allgemeinen lässt dieses iterative Vorgehen mehr Fehler zu, was jedoch dazu führen kann, mehr auszuprobieren und zu wagen. Das Buch von Kim Goodwin \glqq{}Designing for the Digital Age\grqq{},  welches stark für die Durchführung des User Centerd Designs zu Rate gezogen wurde, verweist bei der Suche nach Testbenutzern auf professionelle Firmen, die solche Personen stellen. Die Annahme, in der freien Wirtschaft Menschen zu finden, die helfen ein Produkt, das sie selber brauchen zu verbessern, wurde klar widerlegt. Die Wahl der Technologien führte trotz ihrer Neuheit zu keinen erheblichen Problemen. In gewissen Situationen war es schwierig Informationen oder Hilfe für kleinere Probleme beziehungsweise architektonische Fragen zu bekommen. Aber davon abgesehen waren sie die richtige Wahl, um diesen Prototypen damit zu entwickeln. ImagineCargo besitzt nun eine Grundlage, um eine Web Applikation zu entwickeln, mit welcher ein Grossteil der Benutzer zum ersten Mal mit dem Service in Berührung kommen.

